\documentclass[a4paper]{article}
\usepackage[titletoc]{appendix}
\usepackage[margin=1in]{geometry}

\usepackage{dejavu}
\usepackage[T1]{fontenc}
\usepackage[adobe-utopia]{mathdesign}			



\usepackage[framemethod=tikz]{mdframed}
\usepackage{graphicx}


\usepackage[utf8]{inputenc}
 
\usepackage{listings}
%\usepackage{color}
\usepackage{lipsum}
%\usepackage{courier}

\usepackage{verbatim}


% ---------------------------------------------------------------------
% Syntax Rvised from
% - https://github.com/isagalaev/highlight.js/blob/master/src/languages/stata.js
% - https://github.com/jpitblado/vim-stata/blob/master/syntax/stata.vim
% - http://fmwww.bc.edu/RePEc/bocode/s/synlightlist.ado

\usepackage{listings}



% ---------------------------------------------------------------------
% Stata language definition

\lstdefinelanguage{stata}{
  sensitive=true,
  %
  % Macros, global and local
  alsoletter={\{\}0123456789},
  keywordsprefix=\$,
  morecomment=[n][keywordstyle9]{`}{'},
  morekeywords={},
  %
  % Comments
  morecomment=[f][\color{Green}\slshape][0]*,
  morecomment=[l]{//},
  morecomment=[s]{/*}{*/},
  %
  % Strings
  morecomment=[n][\color{Maroon}]{`"}{"'},
  morestring=[b]",
  %
  % Add-ons and system Commands
  morekeywords=[2]{
    if ,else ,in ,foreach ,forv ,forva ,forval ,forvalu ,forvalue
    ,forvalues ,by ,bys ,bysort ,xi ,quietly ,qui ,capture ,about
    ,ac ,ac_7 ,acprplot ,acprplot_7 adjust ,ado ,adopath ,adoupdate
    ,alpha ,ameans ,an ,ano ,anov ,anova ,anova_estat ,anova_terms
    ,anovadef ,aorder ,ap ,app ,appe ,appen ,append ,arch ,arch_dr
    ,arch_estat ,arch_p ,archlm ,areg ,areg_p ,args ,arima ,arima_dr
    ,arima_estat ,arima_p ,as ,asmprobit ,asmprobit_estat ,asmprobit_lf
    ,asmprobit_mfx__dlg ,asmprobit_p ,ass ,asse ,asser ,assert ,avplot
    ,avplot_7 ,avplots ,avplots_7 bcskew0 ,bgodfrey ,binreg ,bip0_lf
    ,biplot ,bipp_lf ,bipr_lf ,bipr_p ,biprobit ,bitest ,bitesti
    ,bitowt ,blogit ,bmemsize ,boot ,bootsamp ,bootstrap ,bootstrap_8
    ,boxco_l ,boxco_p ,boxcox ,boxcox_6 ,boxcox_p ,bprobit ,br ,break
    ,brier ,bro ,brow ,brows ,browse ,brr ,brrstat ,bs ,bs_7 ,bsampl_w
    ,bsample ,bsample_7 ,bsqreg ,bstat ,bstat_7 ,bstat_8 ,bstrap
    ,bstrap_7 ,ca ,ca_estat ,ca_p ,cabiplot ,camat ,canon ,canon_8
    ,canon_8_p ,canon_estat ,canon_p ,cap ,caprojection ,capt ,captu
    ,captur ,capture ,cat ,cc ,cchart ,cchart_7 ,cci ,cd ,censobs_table
    ,centile ,cf ,char ,chdir ,checkdlgfiles ,checkestimationsample
    ,checkhlpfiles ,checksum ,chelp ,ci ,cii ,cl ,class ,classutil
    ,clear ,cli ,clis ,clist ,clo ,clog ,clog_lf ,clog_p ,clogi
    ,clogi_sw ,clogit ,clogit_lf ,clogit_p ,clogitp ,clogl_sw ,cloglog
    ,clonevar ,clslistarray ,cluster ,cluster_measures ,cluster_stop
    ,cluster_tree ,cluster_tree_8 ,clustermat ,cmdlog ,cnr ,cnre
    ,cnreg ,cnreg_p ,cnreg_sw ,cnsreg ,codebook ,collaps4 ,collapse
    ,colormult_nb ,colormult_nw ,compare ,compress ,conf ,confi
    ,confir ,confirm ,conren ,cons ,const ,constr ,constra ,constrai
    ,constrain ,constraint ,continue ,contract ,copy ,copyright
    ,copysource ,cor ,corc ,corr ,corr2data ,corr_anti ,corr_kmo
    ,corr_smc ,corre ,correl ,correla ,correlat ,correlate ,corrgram
    ,cou ,coun ,count ,cox ,cox_p ,cox_sw ,coxbase ,coxhaz ,coxvar
    ,cprplot ,cprplot_7 ,crc ,cret ,cretu ,cretur ,creturn ,cross ,cs
    ,cscript ,cscript_log ,csi ,ct ,ct_is ,ctset ,ctst_5 ,ctst_st
    ,cttost ,cumsp ,cumsp_7 ,cumul ,cusum ,cusum_7 ,cutil ,d ,datasig
    ,datasign ,datasigna ,datasignat ,datasignatu ,datasignatur
    ,datasignature ,datetof ,db ,dbeta ,de ,dec ,deco ,decod ,decode
    ,deff ,delim ,des ,desc ,descr ,descri ,describ ,describe ,destring
    ,dfbeta ,dfgls ,dfuller ,di ,di_g ,dir ,dirstats ,dis ,discard
    ,disp ,disp_res ,disp_s ,displ ,displa ,display ,distinct ,do
    ,doe ,doed ,doedi ,doedit ,dotplot ,dotplot_7 ,dprobit ,drawnorm
    ,drop ,ds ,ds_util ,dstdize ,duplicates ,durbina ,dwstat ,dydx ,e
    ,ed ,end ,edi ,edit ,egen ,eivreg ,emdef ,en ,enc ,enco ,encod ,encode
    ,eq ,erase ,ereg ,ereg_lf ,ereg_p ,ereg_sw ,ereghet ,ereghet_glf
    ,ereghet_glf_sh ,ereghet_gp ,ereghet_ilf ,ereghet_ilf_sh ,ereghet_ip
    ,eret ,eretu ,eretur ,ereturn ,err ,erro ,error ,est ,est_cfexist
    ,est_cfname ,est_clickable ,est_expand ,est_hold ,est_table
    ,est_unhold ,est_unholdok ,estat ,estat_default ,estat_summ
    ,estat_vce_only ,esti ,estimates ,etodow ,etof ,etomdy ,ex ,exi
    ,exit ,expand ,expandcl ,fac ,fact ,facto ,factor ,factor_estat
    ,factor_p ,factor_pca_rotated ,factor_rotate ,factormat ,fcast
    ,fcast_compute ,fcast_graph ,fdades ,fdadesc ,fdadescr ,fdadescri
    ,fdadescrib ,fdadescribe ,fdasav ,fdasave ,fdause ,fh_st
    ,open ,read ,close ,filefilter ,fillin
    ,find_hlp_file ,findfile ,findit ,findit_7 ,fit ,fl ,fli ,flis
    ,flist ,for5_0 ,form ,forma ,format ,fpredict ,frac_154 ,frac_adj
    ,frac_chk ,frac_cox ,frac_ddp ,frac_dis ,frac_dv ,frac_in ,frac_mun
    ,frac_pp ,frac_pq ,frac_pv ,frac_wgt ,frac_xo ,fracgen ,fracplot
    ,fracplot_7 ,fracpoly ,fracpred ,fron_ex ,fron_hn ,fron_p ,fron_tn
    ,fron_tn2 ,frontier ,ftodate ,ftoe ,ftomdy ,ftowdate ,g ,gamhet_glf
    ,gamhet_gp ,gamhet_ilf ,gamhet_ip ,gamma ,gamma_d2 ,gamma_p
    ,gamma_sw ,gammahet ,gdi_hexagon ,gdi_spokes ,ge ,gen ,gene ,gener
    ,genera ,generat ,generate ,genrank ,genstd ,genvmean ,gettoken
    ,gl ,gladder ,gladder_7 ,glim_l01 ,glim_l02 glim_l03 ,glim_l04
    ,glim_l05 ,glim_l06 ,glim_l07 ,glim_l08 ,glim_l09 ,glim_l10 glim_l11
    ,glim_l12 ,glim_lf ,glim_mu ,glim_nw1 ,glim_nw2 ,glim_nw3 ,glim_p
    ,glim_v1 ,glim_v2 ,glim_v3 ,glim_v4 ,glim_v5 ,glim_v6 ,glim_v7 ,glm
    ,glm_6 glm_p ,glm_sw ,glmpred ,glo ,glob ,globa ,global ,glogit
    ,glogit_8 ,glogit_p ,gmeans ,gnbre_lf ,gnbreg ,gnbreg_5 ,gnbreg_p
    ,gomp_lf ,gompe_sw ,gomper_p ,gompertz ,gompertzhet ,gomphet_glf
    ,gomphet_glf_sh ,gomphet_gp ,gomphet_ilf ,gomphet_ilf_sh ,gomphet_ip
    ,gphdot ,gphpen ,gphprint ,gprefs ,gprobi_p ,gprobit ,gprobit_8
    ,gr ,gr7 ,gr_copy ,gr_current ,gr_db ,gr_describe ,gr_dir ,gr_draw
    ,gr_draw_replay ,gr_drop ,gr_edit ,gr_editviewopts ,gr_example
    ,gr_example2 gr_export ,gr_print ,gr_qscheme ,gr_query ,gr_read
    ,gr_rename ,gr_replay ,gr_save ,gr_set ,gr_setscheme ,gr_table
    ,gr_undo ,gr_use ,graph ,graph7 grebar ,greigen ,greigen_7
    ,greigen_8 ,grmeanby ,grmeanby_7 ,gs_fileinfo ,gs_filetype
    ,gs_graphinfo ,gs_stat ,gsort ,gwood ,h ,hadimvo ,hareg ,hausman
    ,haver ,he ,heck_d2 ,heckma_p ,heckman ,heckp_lf ,heckpr_p ,heckprob
    ,hel ,help ,hereg ,hetpr_lf ,hetpr_p ,hetprob ,hettest ,hexdump
    ,hilite ,hist ,hist_7 histogram ,hlogit ,hlu ,hmeans ,hotel
    ,hotelling ,hprobit ,hreg ,hsearch ,icd9 ,icd9_ff ,icd9p ,iis
    ,impute ,imtest ,import ,inbase ,include ,inf ,infi ,infil ,infile ,infix
    ,inp ,inpu ,input ,ins ,insheet ,insp ,inspe ,inspec ,inspect ,integ
    ,inten ,intreg ,intreg_7 ,intreg_p ,intrg2_ll ,intrg_ll ,intrg_ll2
    ,ipolate ,iqreg ,ir ,irf ,irf_create ,irfm ,iri ,is_svy ,is_svysum
    ,isid ,istdize ,ivprob_1_lf ,ivprob_lf ,ivprobit ,ivprobit_p ,ivreg
    ,ivreg_footnote ,ivtob_1_lf ,ivtob_lf ,ivtobit ,ivtobit_p ,jackknife
    ,jacknife ,jknife ,jknife_6 ,jknife_8 ,jkstat ,joinby ,kalarma1
    ,kap ,kap_3 ,kapmeier ,kappa ,kapwgt ,kdensity ,kdensity_7 keep
    ,ksm ,ksmirnov ,ktau ,kwallis ,l ,la ,lab ,labe ,label ,labelbook
    ,ladder ,levels ,levelsof ,leverage ,lfit ,lfit_p ,li ,lincom ,line
    ,linktest ,lis ,list ,lloghet_glf ,lloghet_glf_sh ,lloghet_gp
    ,lloghet_ilf ,lloghet_ilf_sh ,lloghet_ip ,llogi_sw ,llogis_p
    ,llogist ,llogistic ,llogistichet ,lnorm_lf ,lnorm_sw ,lnorma_p
    ,lnormal ,lnormalhet ,lnormhet_glf ,lnormhet_glf_sh ,lnormhet_gp
    ,lnormhet_ilf ,lnormhet_ilf_sh ,lnormhet_ip ,lnskew0 ,loadingplot
    ,loc ,loca ,local ,logi ,logis_lf ,logistic ,logistic_p
    ,logit ,logit_estat ,logit_p ,loglogs ,logrank ,loneway ,lookfor
    ,lookup ,lowess ,lowess_7 ,lpredict ,lrecomp ,lroc ,lroc_7 ,lrtest
    ,ls ,lsens ,lsens_7 ,lsens_x ,lstat ,ltable ,ltable_7 ,ltriang
    ,lv ,lvr2plot ,lvr2plot_7 ,m ,ma ,mac ,macr ,macro ,makecns ,man
    ,manova ,manova_estat ,manova_p ,manovatest ,mantel ,mark ,markin
    ,markout ,marksample ,mat ,mat_capp ,mat_order ,mat_put_rr ,mat_rapp
    ,mata ,mata_clear ,mata_describe ,mata_drop ,mata_matdescribe
    ,mata_matsave ,mata_matuse ,mata_memory ,mata_mlib ,mata_mosave
    ,mata_rename ,mata_which ,matalabel ,matcproc ,matlist ,matname
    ,matr ,matri ,matrix ,matrix_input__dlg ,matstrik ,mcc ,mcci ,md0_
    ,md1_ ,md1debug_ ,md2_ ,md2debug_ ,mds ,mds_estat ,mds_p ,mdsconfig
    ,mdslong ,mdsmat ,mdsshepard ,mdytoe ,mdytof ,me_derd ,mean ,means
    ,median ,memory ,memsize ,meqparse ,mer ,merg ,merge ,mfp ,mfx
    ,mhelp ,mhodds ,minbound ,mixed_ll ,mixed_ll_reparm ,mkassert
    ,mkdir ,mkmat ,mkspline ,ml ,ml_5 ml_adjs ,ml_bhhhs ,ml_c_d
    ,ml_check ,ml_clear ,ml_cnt ,ml_debug ,ml_defd ,ml_e0 ml_e0_bfgs
    ,ml_e0_cycle ,ml_e0_dfp ,ml_e0i ,ml_e1 ,ml_e1_bfgs ,ml_e1_bhhh
    ,ml_e1_cycle ,ml_e1_dfp ,ml_e2 ,ml_e2_cycle ,ml_ebfg0 ,ml_ebfr0
    ,ml_ebfr1 ml_ebh0q ,ml_ebhh0 ,ml_ebhr0 ,ml_ebr0i ,ml_ecr0i ,ml_edfp0
    ,ml_edfr0 ,ml_edfr1 ml_edr0i ,ml_eds ,ml_eer0i ,ml_egr0i ,ml_elf
    ,ml_elf_bfgs ,ml_elf_bhhh ,ml_elf_cycle ,ml_elf_dfp ,ml_elfi
    ,ml_elfs ,ml_enr0i ,ml_enrr0 ,ml_erdu0 ml_erdu0_bfgs ,ml_erdu0_bhhh
    ,ml_erdu0_bhhhq ,ml_erdu0_cycle ,ml_erdu0_dfp ,ml_erdu0_nrbfgs
    ,ml_exde ,ml_footnote ,ml_geqnr ,ml_grad0 ,ml_graph ,ml_hbhhh
    ,ml_hd0 ,ml_hold ,ml_init ,ml_inv ,ml_log ,ml_max ,ml_mlout
    ,ml_mlout_8 ,ml_model ,ml_nb0 ,ml_opt ,ml_p ,ml_plot ,ml_query
    ,ml_rdgrd ,ml_repor ,ml_s_e ,ml_score ,ml_searc ,ml_technique
    ,ml_unhold ,mleval ,mlf_ ,mlmatbysum ,mlmatsum ,mlog ,mlogi ,mlogit
    ,mlogit_footnote ,mlogit_p ,mlopts ,mlsum ,mlvecsum ,mnl0_ ,mor
    ,more ,mov ,move ,mprobit ,mprobit_lf ,mprobit_p ,mrdu0_ ,mrdu1_
    ,mvdecode ,mvencode ,mvreg ,mvreg_estat ,nbreg ,nbreg_al
    ,nbreg_lf ,nbreg_p ,nbreg_sw ,nestreg ,net ,newey ,newey_7 ,newey_p
    ,news ,nl ,nl_7 ,nl_9 ,nl_9_p ,nl_p ,nl_p_7 nlcom ,nlcom_p ,nlexp2
    ,nlexp2_7 ,nlexp2a ,nlexp2a_7 ,nlexp3 ,nlexp3_7 ,nlgom3 nlgom3_7
    ,nlgom4 ,nlgom4_7 ,nlinit ,nllog3 ,nllog3_7 ,nllog4 ,nllog4_7
    ,nlog_rd ,nlogit ,nlogit_p ,nlogitgen ,nlogittree ,nlpred ,no
    ,nobreak ,noi ,nois ,noisi ,noisil ,noisily ,note ,notes ,notes_dlg
    ,nptrend ,numlabel ,numlist ,odbc ,old_ver ,olo ,olog ,ologi
    ,ologi_sw ,ologit ,ologit_p ,ologitp ,on ,onew ,onewa ,oneway
    ,op_colnm ,op_comp ,op_diff ,op_inv ,op_str ,opr ,opro ,oprob
    ,oprob_sw ,oprobi ,oprobi_p ,oprobit ,oprobitp ,opts_exclusive
    ,order ,orthog ,orthpoly ,ou ,out ,outf ,outfi ,outfil ,outfile
    ,outs ,outsh ,outshe ,outshee ,outsheet ,ovtest ,pac ,pac_7 ,palette
    ,parse ,parse_dissim ,pause ,pca ,pca_8 pca_display ,pca_estat
    ,pca_p ,pca_rotate ,pcamat ,pchart ,pchart_7 ,pchi ,pchi_7 ,pcorr
    ,pctile ,pentium ,pergram ,pergram_7 ,permute ,permute_8 ,personal
    ,peto_st ,pkcollapse ,pkcross ,pkequiv ,pkexamine ,pkexamine_7
    ,pkshape ,pksumm ,pksumm_7 ,pl ,plo ,plot ,plugin ,pnorm ,pnorm_7
    ,poisgof ,poiss_lf ,poiss_sw ,poisso_p ,poisson ,poisson_estat
    ,post ,postclose ,postfile ,postutil ,pperron ,pr ,prais ,prais_e
    ,prais_e2 ,prais_p ,predict ,predictnl ,preserve ,print ,pro ,prob
    ,probi ,probit ,probit_estat ,probit_p ,proc_time ,procoverlay
    ,procrustes ,procrustes_estat ,procrustes_p ,profiler ,prog ,progr
    ,progra ,program ,prop ,proportion ,prtest ,prtesti ,pwcorr ,pwd
    ,q ,s ,qby ,qbys ,qchi ,qchi_7 ,qladder ,qladder_7 ,qnorm ,qnorm_7
    ,qqplot ,qqplot_7 ,qreg ,qreg_c ,qreg_p ,qreg_sw ,qu ,quadchk
    ,quantile ,quantile_7 ,que ,quer ,query ,range ,ranksum ,ratio
    ,rchart ,rchart_7 ,rcof ,recast ,reclink ,recode ,reg ,reg3
    ,reg3_p ,regdw ,regr ,regre ,regre_p2 ,regres ,regres_p ,regress
    ,regress_estat ,regriv_p ,remap ,ren ,rena ,renam ,rename ,renpfix
    ,repeat ,replace ,report ,reshape ,restore ,ret ,retu ,retur ,return
    ,rm ,rmdir ,robvar ,roccomp ,roccomp_7 ,roccomp_8 ,rocf_lf ,rocfit
    ,rocfit_8 ,rocgold ,rocplot ,rocplot_7 ,roctab ,roctab_7 ,rolling
    ,rologit ,rologit_p ,rot ,rota ,rotat ,rotate ,rotatemat ,rclass ,rreg
    ,rreg_p ,ru ,run ,runtest ,rvfplot ,rvfplot_7 ,rvpplot ,rvpplot_7
    ,sa ,safesum ,sampsi ,sav ,save ,savedresults ,saveold ,sc
    ,sca ,scal ,scala ,scalar ,scatter ,scm_mine ,sco ,scob_lf ,scob_p
    ,scobi_sw ,scobit ,scor ,score ,scoreplot ,scoreplot_help ,scree
    ,screeplot ,screeplot_help ,sdtest ,sdtesti ,se ,seed ,search ,separate
    ,seperate ,serrbar ,serrbar_7 ,serset ,set ,set_defaults ,sfrancia
    ,sh ,she ,shel ,shell ,shewhart ,shewhart_7 ,signestimationsample
    ,signrank ,signtest ,simul ,simul_7 simulate ,simulate_8 ,sktest
    ,sleep ,slogit ,slogit_d2 ,slogit_p ,smooth ,snapspan ,so ,sor
    ,sort ,spearman ,spikeplot ,spikeplot_7 ,spikeplt ,spline_x ,split
    ,sqreg ,sqreg_p ,sret ,sretu ,sretur ,sreturn ,ssc ,st ,st_ct ,st_hc
    ,st_hcd ,st_hcd_sh ,st_is ,st_issys ,st_note ,st_promo ,st_set
    ,st_show ,st_smpl ,st_subid ,stack ,stata ,statsby ,statsby_8 ,stbase
    ,stci ,stci_7 ,stcox ,stcox_estat ,stcox_fr ,stcox_fr_ll ,stcox_p
    ,stcox_sw ,stcoxkm ,stcoxkm_7 ,stcstat ,stcurv ,stcurve ,stcurve_7
    ,stdes ,stem ,stepwise ,stereg ,stfill ,stgen ,stir ,stjoin ,stmc
    ,stmh ,stphplot ,stphplot_7 ,stphtest ,stphtest_7 ,stptime ,strate
    ,strate_7 ,streg ,streg_sw ,streset ,sts ,sts_7 ,stset ,stsplit
    ,stsum ,sttocc ,sttoct ,stvary ,stweib ,su ,suest ,suest_8 ,sum
    ,summ ,summa ,summar ,summari ,summariz ,summarize ,sunflower
    ,sureg ,survcurv ,survsum ,svar ,svar_p ,svmat ,svy ,svy_disp
    ,svy_dreg ,svy_est ,svy_est_7 ,svy_estat ,svy_get ,svy_gnbreg_p
    ,svy_head ,svy_header ,svy_heckman_p ,svy_heckprob_p ,svy_intreg_p
    ,svy_ivreg_p ,svy_logistic_p ,svy_logit_p ,svy_mlogit_p ,svy_nbreg_p
    ,svy_ologit_p ,svy_oprobit_p ,svy_poisson_p ,svy_probit_p
    ,svy_regress_p ,svy_sub ,svy_sub_7 ,svy_x ,svy_x_7 ,svy_x_p ,svydes
    ,svydes_8 ,svygen ,svygnbreg ,svyheckman ,svyheckprob ,svyintreg
    ,svyintreg_7 ,svyintrg ,svyivreg ,svylc ,svylog_p ,svylogit
    ,svymarkout ,svymarkout_8 ,svymean ,svymlog ,svymlogit ,svynbreg
    ,svyolog ,svyologit ,svyoprob ,svyoprobit ,svyopts ,svypois
    ,svypois_7 svypoisson ,svyprobit ,svyprobt ,svyprop ,svyprop_7
    ,svyratio ,svyreg ,svyreg_p ,svyregress ,svyset ,svyset_7 ,svyset_8
    ,svytab ,svytab_7 ,svytest ,svytotal ,sw ,sw_8 ,swcnreg ,swcox
    ,swereg ,swilk ,swlogis ,swlogit ,swologit ,swoprbt ,swpois
    ,swprobit ,swqreg ,swtobit ,swweib ,symmetry ,symmi ,symplot
    ,symplot_7 syntax ,sysdescribe ,sysdir ,sysuse ,szroeter ,ta ,tab
    ,tab1 ,tab2 ,tab_or ,tabd ,tabdi ,tabdis ,tabdisp ,tabi ,table
    ,tabodds ,tabodds_7 ,tabstat ,,tabstatmat ,tabu ,tabul ,tabula ,tabulat ,tabulate
    ,te ,tempfile ,tempname ,tempvar ,tes ,tsrtest ,ritest ,testnl ,testparm
    ,teststd ,tetrachoric ,time_it ,timer ,tis ,tob ,tobi ,tobit
    ,tobit_p ,tobit_sw ,token ,tokeni ,tokeniz ,tokenize ,tostring
    ,total ,translate ,translator ,transmap ,treat_ll ,treatr_p
    ,treatreg ,trim ,trnb_cons ,trnb_mean ,trpoiss_d2 ,trunc_ll
    ,truncr_p ,truncreg ,tsappend ,tset ,tsfill ,tsline ,tsline_ex
    ,tsreport ,tsrevar ,tsrline ,tsset ,tssmooth ,tsunab ,ttest
    ,ttesti ,tut_chk ,tut_wait ,tutorial ,tw ,tware_st ,twoway
    ,twoway__fpfit_serset ,twoway__function_gen ,twoway__histogram_gen
    ,twoway__ipoint_serset ,twoway__ipoints_serset ,twoway__kdensity_gen
    ,twoway__lfit_serset ,twoway__normgen_gen ,twoway__pci_serset
    ,twoway__qfit_serset ,twoway__scatteri_serset ,twoway__sunflower_gen
    ,twoway_ksm_serset ,ty ,typ ,type ,typeof ,u ,unab ,unabbrev
    ,unabcmd ,update ,us ,use ,uselabel ,var ,var_mkcompanion
    ,var_p ,varbasic ,varfcast ,vargranger ,varirf ,varirf_add
    ,varirf_cgraph ,varirf_create ,varirf_ctable ,varirf_describe
    ,varirf_dir ,varirf_drop ,varirf_erase ,varirf_graph ,varirf_ograph
    ,varirf_rename ,varirf_set ,varirf_table ,varlist ,varlmar
    ,varnorm ,varsoc ,varstable ,varstable_w ,varstable_w2 ,varwle
    ,vce ,vec ,vec_fevd ,vec_mkphi ,vec_p ,vec_p_w ,vecirf_create
    ,veclmar ,veclmar_w ,vecnorm ,vecnorm_w ,vecrank ,vecstable
    ,verinst ,vers ,versi ,versio ,version ,view ,viewsource ,vif
    ,vwls ,wdatetof ,webdescribe ,webseek ,webuse ,weib1_lf ,weib2_lf
    ,weib_lf ,weib_lf0 weibhet_glf ,weibhet_glf_sh ,weibhet_glfa
    ,weibhet_glfa_sh ,weibhet_gp ,weibhet_ilf ,weibhet_ilf_sh
    ,weibhet_ilfa ,weibhet_ilfa_sh ,weibhet_ip ,weibu_sw ,weibul_p
    ,weibull ,weibull_c ,weibull_s ,weibullhet ,wh ,whelp ,whi ,which
    ,whil ,while ,wilc_st ,wilcoxon ,win ,wind ,windo ,window ,winexec
    ,wntestb ,wntestb_7 ,wntestq ,xchart ,xchart_7 ,xcorr ,xcorr_7 ,xi
    ,xi_6 ,xmlsav ,xmlsave ,xmluse ,xpose ,xsh ,xshe ,xshel ,xshell
    ,xt_iis ,xt_tis ,xtab_p ,xtabond ,xtbin_p ,xtclog ,xtcloglog
    ,xtcloglog_8 ,xtcloglog_d2 ,xtcloglog_pa_p ,xtcloglog_re_p ,xtcnt_p
    ,xtcorr ,xtdata ,xtdes ,xtfront_p ,xtfrontier ,xtgee ,xtgee_elink
    ,xtgee_estat ,xtgee_makeivar ,xtgee_p ,xtgee_plink ,xtgls ,xtgls_p
    ,xthaus ,xthausman ,xtht_p ,xthtaylor ,xtile ,xtint_p ,xtintreg
    ,xtintreg_8 ,xtintreg_d2 xtintreg_p ,xtivp_1 ,xtivp_2 ,xtivreg
    ,xtline ,xtline_ex ,xtlogit ,xtlogit_8 xtlogit_d2 ,xtlogit_fe_p
    ,xtlogit_pa_p ,xtlogit_re_p ,xtmixed ,xtmixed_estat ,xtmixed_p
    ,xtnb_fe ,xtnb_lf ,xtnbreg ,xtnbreg_pa_p ,xtnbreg_refe_p ,xtpcse
    ,xtpcse_p ,xtpois ,xtpoisson ,xtpoisson_d2 ,xtpoisson_pa_p
    ,xtpoisson_refe_p ,xtpred ,xtprobit ,xtprobit_8 ,xtprobit_d2
    ,xtprobit_re_p ,xtps_fe ,xtps_lf ,xtps_ren ,xtps_ren_8 ,xtrar_p
    ,xtrc ,xtrc_p ,xtrchh ,xtrefe_p ,xtreg ,xtreg_be ,xtreg_fe
    ,xtreg_ml ,xtreg_pa_p ,xtreg_re ,xtregar ,xtrere_p ,xtset
    ,xtsf_ll ,xtsf_llti ,xtsum ,xttab ,xttest0 ,xttobit ,xttobit_8
    ,xttobit_p ,xttrans ,yx ,yxview__barlike_draw ,yxview_area_draw
    ,yxview_bar_draw ,yxview_dot_draw ,yxview_dropline_draw
    ,yxview_function_draw ,yxview_iarrow_draw ,yxview_ilabels_draw
    ,yxview_normal_draw ,yxview_pcarrow_draw ,yxview_pcbarrow_draw
    ,yxview_pccapsym_draw ,yxview_pcscatter_draw ,yxview_pcspike_draw
    ,yxview_rarea_draw ,yxview_rbar_draw ,yxview_rbarm_draw
    ,yxview_rcap_draw ,yxview_rcapsym_draw ,yxview_rconnected_draw
    ,yxview_rline_draw ,yxview_rscatter_draw ,yxview_rspike_draw
    ,yxview_spike_draw ,yxview_sunflower_draw ,zap_s ,zinb ,zinb_llf
    ,zinb_plf ,zip ,zip_llf ,zip_p ,zip_plf ,zt_ct_5 ,zt_hc_5 ,zt_hcd_5
    ,zt_is_5 ,zt_iss_5 ,zt_sho_5 zt_smp_5 ,ztbase_5 ,ztcox_5 ,ztdes_5
    ,ztereg_5 ,ztfill_5 ,ztgen_5 ,ztir_5 ztjoin_5 ,ztnb ,ztnb_p ,ztp
    ,ztp_p ,zts_5 ,ztset_5 ,ztspli_5 ,ztsum_5 ,zttoct_5 ztvary_5
    ,ztweib_5
  },
  %
  % Built-in functions
  morekeywords=[3]{
    Cdhms ,Chms ,Clock ,Cmdyhms ,Cofc ,Cofd ,F ,Fden ,Ftail ,I ,J
    ,_caller ,abbrev ,abs ,acos ,acosh ,asin ,asinh ,atan ,atan2
    ,atanh ,autocode ,betaden ,binomial ,binomialp ,binomialtail
    ,binormal ,bofd ,byteorder ,ceil ,char ,chi2 ,chi2den ,chi2tail
    ,cholesky ,chop ,clip ,clock ,cloglog ,cofC ,cofd ,colnumb ,colsof
    ,comb ,cond ,corr ,cos ,cosh ,d ,daily ,date ,day ,det ,dgammapda
    ,dgammapdada ,dgammapdadx ,dgammapdx ,dgammapdxdx ,dhms ,diag
    ,diag0cnt ,digamma ,dofC ,dofb ,dofc ,dofh ,dofm ,dofq ,dofw ,dofy
    ,dow ,doy ,dunnettprob ,e ,el ,epsdouble ,epsfloat ,exp ,fileexists
    ,fileread ,filereaderror ,filewrite ,float ,floor ,fmtwidth
    ,gammaden ,gammap ,gammaptail ,get ,group ,h ,hadamard ,halfyear
    ,halfyearly ,has_eprop ,hh ,hhC ,hms ,hofd ,hours ,hypergeometric
    ,hypergeometricp ,ibeta ,ibetatail ,index ,indexnot ,inlist
    ,inrange ,int ,inv ,invF ,invFtail ,invbinomial ,invbinomialtail
    ,invchi2 ,invchi2tail ,invcloglog ,invdunnettprob ,invgammap
    ,invgammaptail ,invibeta ,invibetatail ,invlogit ,invnFtail
    ,invnbinomial ,invnbinomialtail ,invnchi2 ,invnchi2tail ,invnibeta
    ,invnorm ,invnormal ,invnttail ,invpoisson ,invpoissontail ,invsym
    ,invt ,invttail ,invtukeyprob ,irecode ,issym ,issymmetric ,itrim
    ,length ,ln ,lnfact ,lnfactorial ,lngamma ,lnnormal ,lnnormalden
    ,log10 ,logit ,lower ,ltrim ,m ,match ,matmissing ,matrix
    ,matuniform ,max ,maxbyte ,maxdouble ,maxfloat ,maxint ,maxlong ,mdy
    ,mdyhms ,mi ,mi ,min ,minbyte ,mindouble ,minfloat ,minint ,minlong
    ,minutes ,mm ,mmC ,mod ,mofd ,month ,monthly ,mreldif
    ,msofhours ,msofminutes ,msofseconds ,nF ,nFden ,nFtail ,nbetaden
    ,nbinomial ,nbinomialp ,nbinomialtail ,nchi2 ,nchi2den ,nchi2tail
    ,nibeta ,norm ,normal ,normalden ,normd ,npnF ,npnchi2 ,npnt ,nt
    ,ntden ,nttail ,nullmat ,plural ,poisson ,poissonp ,poissontail
    ,proper ,q ,qofd ,quarter ,quarterly ,r ,rbeta ,rbinomial ,rchi2
    real ,recode ,regexm ,regexr ,regexs ,reldif ,replay ,return
    ,reverse ,rgamma ,rhypergeometric ,rnbinomial ,rnormal ,round
    ,rownumb ,rowsof ,rpoisson ,rt ,rtrim ,runiform ,s ,scalar ,seconds
    ,sign ,sin ,sinh ,smallestdouble ,soundex ,soundex_nara ,sqrt ,ss
    ,ssC ,strcat ,strdup ,string ,strlen ,strlower ,strltrim ,strmatch
    ,strofreal ,strpos ,strproper ,strreverse ,strrtrim ,strtoname
    ,strtrim ,strupper ,subinstr ,subinword ,substr ,sum ,sweep ,syminv
    ,t ,tC ,tan ,tanh ,tc ,td ,tden ,th ,tin ,tm ,tq ,trace ,trigamma
    ,trim ,trunc ,ttail ,tukeyprob ,tw ,twithin ,uniform ,upper ,vec
    ,vecdiag ,w ,week ,weekly ,wofd ,word ,wordcount ,year ,yearly
    ,yh ,ym ,yofd ,yq ,yw
  },
}

% ---------------------------------------------------------------------
% Stata highligh style

\providecommand{\textcolordummy}[2]{#2}
\lstalias{Stata}{stata}





 
%\definecolor{codegreen}{rgb}{0,0.6,0}
%\definecolor{codegray}{rgb}{0.5,0.5,0.5}
\definecolor{codepurple}{rgb}{0.70,0.20,0.60}
\definecolor{backcolour}{rgb}{0.97,0.97,0.97}
\definecolor{basiccode}{rgb}{0.18,0.18,0.18}
\definecolor{ao(english)}{rgb}{0.0, 0.5, 0.0}
\definecolor{byzantium}{rgb}{0.44, 0.16, 0.39}
 %\definecolor{darkmagenta}{rgb}{0.55, 0.0, 0.55}
 %\definecolor{darkorchid}{rgb}{0.6, 0.2, 0.8}
 %\definecolor{falured}{rgb}{0.5, 0.09, 0.09}
 \definecolor{brightmaroon}{rgb}{0.76, 0.13, 0.28}
 
 
\lstdefinestyle{mystyle}{
    backgroundcolor=\color{backcolour},   
    commentstyle=\color{ao(english)},
    keywordstyle=\color{brightmaroon},
    %numberstyle=\tiny\color{codegray},
    stringstyle=\color{codepurple},
    breakatwhitespace=false,         
    breaklines=true,                 
    captionpos=b,                    
    keepspaces=false,                                  
    numbersep=5pt,                  
    showspaces=false,                
    showstringspaces=false,
    showtabs=false,                  
    basicstyle= \footnotesize\ttfamily\color{basiccode},
    tabsize = 2
}
 
\lstset{style=mystyle}




\begin{document}

\title{Stata Code and Results Sample}
\author{Use Chapter 3 Results }
\date{\today}
\maketitle



\section*{Question 6}
\begin{lstlisting}[language=stata]
. // download data from: http://hdl.handle.net/10079/6hdr852
. // copy and paste the url to your web browser
. 
. import delim "Clingingsmith_et_al_QJE_2009dta.csv",clear
(8 vars, 958 obs)

. set seed 1234567
. rename success D
. rename views Y
//findit tsrtest
. //package name:  st0158.pkg install
. 
. cap program drop ate
. program define ate, rclass
  1.         args Y D
  2.     sum `Y' if `D'==1, meanonly
  3.     local Y_treat=r(mean)
  4.     sum `Y' if `D'==0, meanonly
  5.     local Y_con=r(mean)
  6.     return scalar ate_avg = `Y_treat'-`Y_con'
  7. end

. // ssc install tsrtest
. tsrtest D r(ate_avg) using 3_6_resam.dta, overwrite: ate Y D
Two-sample randomization test for theta=r(ate_avg) of ate Y D by D

Combinations:   8.4503047638e+285 = (958 choose 448)
Assuming null=0
Observed theta: .4748

Minimum time needed for exact test (h:m:s):  4.2e+278:00:00
Reverting to Monte Carlo simulation.
Mode: simulation (10000 repetitions)

progress: |........................................|

 p=0.00190 [one-tailed test of Ho:  theta(D==0)<=theta(D==1)]
 p=0.99830 [one-tailed test of Ho:  theta(D==0)>=theta(D==1)]
 p=0.00360 [two-tailed test of Ho:  theta(D==0)==theta(D==1)]

Saving log file to 3_6_resam.dta...done.
.   
. preserve 
. use "3_6_resam.dta", clear

. global ate = theta[1]

. di $ate
.4748337

. drop if _n==1
(1 observation deleted)

. count if theta >= $ate
  19

. scalar p_onesided = r(N)/_N

. count if abs(theta) >= $ate
  36

. scalar p_twosided = r(N)/_N

. di "p.value.onesided = "p_onesided
p.value.onesided = .0019

. di "p.value.twosided = "p_twosided 
p.value.twosided = .0036

. restore
\end{lstlisting}



\section*{Question 7}
\begin{lstlisting}[language=stata]
. clear
. set seed 1234567
. set obs 10
number of observations (_N) was 0, now 10
. 
. input D Y

             D          Y
  1. 0 1 
  2. 0 0 
  3. 0 0 
  4. 0 4 
  5. 0 3 
  6. 1 2 
  7. 1 11 
  8. 1 14 
  9. 1 0 
 10. 1 3 
. 
. gen Y_star= Y+D*(-7)
. 
. cap program drop ate

. program define ate, rclass
  1.         args Y D
  2.         sum `Y' if `D'==1, meanonly
  3.         local Y_treat=r(mean)
  4.         sum `Y' if `D'==0, meanonly
  5.         local Y_con=r(mean)
  6.         return scalar ate_avg = `Y_treat'-`Y_con'
  7. end
. 
. // findit tsrtest (to install the package)
. tsrtest D r(ate_avg): ate Y_star D
Two-sample randomization test for theta=r(ate_avg) of ate Y_star D by D

Combinations:   252 = (10 choose 5)
Assuming null=0
Observed theta: -2.6

Minimum time needed for exact test (h:m:s):  0:00:00
Mode: exact

progress: |........................................|

 p=0.83730 [one-tailed test of Ho:  theta(D==0)<=theta(D==1)]
 p=0.20635 [one-tailed test of Ho:  theta(D==0)>=theta(D==1)]
 p=0.41270 [two-tailed test of Ho:  theta(D==0)==theta(D==1)]
. 
. // ate
. di r(obsvStat)       
-2.6

. 
. // p.value.onesided
. di r(lowertail)   
.20634921

\end{lstlisting}

\section*{Question 8}
\subsection*{part(a)}
\begin{lstlisting}[language=stata]
 // download data from : http://hdl.handle.net/10079/s1rn910
. // copy and paste the url to your web browser
. use "Titiunik_WorkingPaper_2010.csv.dta",clear 
. 
. set seed 1234567

.         rename term2year D
.         rename bills_introduced Y
.         rename texas0_arkansas1 block         
.         qui tabstat Y if block ==0, by(D) stat(mean) save       
.         scalar ate_texas = el(r(Stat2),1,1) - el(r(Stat1),1,1)         
.         qui tabstat Y if block ==1, by(D) stat(mean) save       
.         scalar ate_ark = el(r(Stat2),1,1) - el(r(Stat1),1,1)
.         
.         di "ate_texas="%18.5f ate_texas 
ate_texas=         -16.74167

.         di "ate_arkansas="%18.5f ate_ark        
ate_arkansas=         -10.09477

\end{lstlisting}

\subsection*{part(b)}
\begin{lstlisting}[language=stata]

 qui tabstat Y if block ==0, by(D) stat(v n) save        
. scalar se_texas = sqrt(el(r(Stat2),1,1)/el(r(Stat2),2,1) + /// 
>                                         el(r(Stat1),1,1)/el(r(Stat1),2,1))
.                                         
. 
. qui tabstat Y if block ==1, by(D) stat(v n) save        

. 
. scalar se_arkansas = sqrt(el(r(Stat2),1,1)/el(r(Stat2),2,1) + /// 
>                                         el(r(Stat1),1,1)/el(r(Stat1),2,1)) 
. 
. di "se_texas="%18.6f se_texas
se_texas=          9.345871

. di "se_arkansas="%18.6f se_arkansas
se_arkansas=          3.395979

\end{lstlisting}

\subsection*{part(c)}
\begin{lstlisting}[language=stata]
 qui tabstat Y, by(block) stat(n) save   
. 
. scalar ate_overall = el(r(Stat1),1,1)/_N*ate_texas + /// 
>                                          el(r(Stat2),1,1)/_N*ate_ark
. 
. 
. di %18.4f ate_overall
          -13.2168
. 
. // same as
. // teffects nnmatch (bills_introduced) (term2year), ematch(texas0_arkansas1)

\end{lstlisting}


\subsection*{part(e)}
\begin{lstlisting}[language=stata]
. scalar se_overall = sqrt((el(r(Stat1),1,1)/_N)^2*se_texas^2 + /// 
>                                          (el(r(Stat2),1,1)/_N)^2*se_arkansas^2)
.                                          
. di %18.5f se_overall
           4.74478

\end{lstlisting}

\subsection*{part(f)}
\begin{lstlisting}[language=stata]
 // calculate probs under block assignment
. bysort block: egen probs=mean(D). 
. 
. 
. cap program drop ate_block

. 
. program define ate_block, rclass
  1. args Y D probs
  2. tempvar ipw
  3. gen `ipw' = .
  4. // calculate inverse probability weight under block assignment
. replace `ipw' = `D'/`probs' + (1-`D')/(1-`probs')
  5. qui reg `Y' `D' [iw=`ipw']
  6. return scalar ate=_b[`D']
  7. end 
. 
. // ssc install ritest (to install ritest package)
. 
. //
. ritest D r(ate), strata(block) reps(10000) nodots: ///
> ate_block Y D probs
(66 missing values generated)
(66 real changes made)

      command:  ate_block Y D probs
        _pm_1:  r(ate)
  res. var(s):  D
   Resampling:  Permuting D
Clust. var(s):  __000000
     Clusters:  66
Strata var(s):  block
       Strata:  2

------------------------------------------------------------------------------
T            |     T(obs)       c       n   p=c/n   SE(p) [95% Conf. Interval]
-------------+----------------------------------------------------------------
       _pm_1 |   -13.2168      65   10000  0.0065  0.0008    .00502   .0082774
------------------------------------------------------------------------------
Note: Confidence interval is with respect to p=c/n.
Note: c = #{|T| >= |T(obs)|}

. // ate
. di el(r(b),1,1)
-13.216796

. 
. // p.value.twosided
. di el(r(p),1,1)
.0065
\end{lstlisting}


\section*{Question 9}
\subsection*{part(b)}

\begin{lstlisting}[language=stata]
 // download data from : http://hdl.handle.net/10079/1g1jx43
. // copy and paste the url to your web browser
. 
. use "Camerer_JPEsubset_1998.dta.dta", clear 
. 
. set seed 1234567

.         rename treatment D

.         rename pair block

.         rename preexperimentbets covs
. 
.         // calculate probs under block assignment
.         bysort block: egen probs=mean(D)

.         
.                 
.         // permuation to calculate F stat and one-side P value
.         ritest D e(F), strata(block) reps(10000) right nodots: ///
>         regress D covs

      Source |       SS           df       MS      Number of obs   =        34
-------------+----------------------------------   F(1, 32)        =      0.02
       Model |  .005024372         1  .005024372   Prob > F        =    0.8914
    Residual |  8.49497563        32  .265467988   R-squared       =    0.0006
-------------+----------------------------------   Adj R-squared   =   -0.0306
       Total |         8.5        33  .257575758   Root MSE        =    .51524

------------------------------------------------------------------------------
           D |      Coef.   Std. Err.      t    P>|t|     [95% Conf. Interval]
-------------+----------------------------------------------------------------
        covs |  -.0000386   .0002809    -0.14   0.891    -.0006109    .0005336
       _cons |   .5137818   .1335793     3.85   0.001     .2416896     .785874
------------------------------------------------------------------------------

      command:  regress D covs
        _pm_1:  e(F)
  res. var(s):  D
   Resampling:  Permuting D
Clust. var(s):  __000000
     Clusters:  34
Strata var(s):  block
       Strata:  17

------------------------------------------------------------------------------
T            |     T(obs)       c       n   p=c/n   SE(p) [95% Conf. Interval]
-------------+----------------------------------------------------------------
       _pm_1 |   .0189265    3736   10000  0.3736  0.0048  .3641064   .3831672
------------------------------------------------------------------------------
Note: Confidence interval is with respect to p=c/n.
Note: c = #{T >= T(obs)}

. 
.         // p.value
.         di el(r(p),1,1)
.3736



\end{lstlisting}

\subsection*{part(c)}
\begin{lstlisting}[language=stata]
. rename experimentbets change

. 
. tabstat change, by(D) stat(mean) save   

Summary for variables: change
     by categories of: D 

       D |      mean
---------+----------
       0 |  571.4118
       1 |  461.2353
---------+----------
   Total |  516.3235
--------------------

. 
. di "ATE ="%180.4f el(r(Stat2),1,1)-el(r(Stat1),1,1)
ATE =                 -110.1765


\end{lstlisting}
\subsection*{part(d)}
\begin{lstlisting}[language=stata]
 bysort block (D): gen pair_diff = change - change[_n+1]
(17 missing values generated)

. mean(pair_diff)

Mean estimation                   Number of obs   =         17

--------------------------------------------------------------
             |       Mean   Std. Err.     [95% Conf. Interval]
-------------+------------------------------------------------
   pair_diff |   110.1765   104.8377     -112.0695    332.4225
--------------------------------------------------------------

. 
. // the same as
. // teffects nnmatch (experimentbets block) (D)

\end{lstlisting}
\subsection*{part(e)}
\begin{lstlisting}[language=stata]

. cap program drop ate_block

. 
. program define ate_block, rclass
  1. args Y D probs
  2. tempvar ipw
  3. gen `ipw' = .
  4. // calculate inverse probability weight under block assignment
. replace `ipw' = `D'/`probs' + (1-`D')/(1-`probs')
  5. qui reg `Y' `D' [iw=`ipw']
  6. return scalar ate=_b[`D']
  7. end 

. 
. 
. ritest D r(ate), strata(block) reps(10000) nodots: ///
> ate_block change D probs
(34 missing values generated)
(34 real changes made)

      command:  ate_block change D probs
        _pm_1:  r(ate)
  res. var(s):  D
   Resampling:  Permuting D
Clust. var(s):  __000000
     Clusters:  34
Strata var(s):  block
       Strata:  17

------------------------------------------------------------------------------
T            |     T(obs)       c       n   p=c/n   SE(p) [95% Conf. Interval]
-------------+----------------------------------------------------------------
       _pm_1 |  -110.1765    3170   10000  0.3170  0.0047  .3078845   .3262222
------------------------------------------------------------------------------
Note: Confidence interval is with respect to p=c/n.
Note: c = #{|T| >= |T(obs)|}. 
. 
. // ate
. di el(r(b),1,1)
-110.17647

. 
. // p.value.twosided
. di el(r(p),1,1)
.317
\end{lstlisting}

\section*{Question 10}
\subsection*{part(a)}
\begin{lstlisting}[language=stata]
. clear

. set seed 1234567

. set obs 14
number of observations (_N) was 0, now 14

. input Y0

            Y0
  1. 0
  2. 1
  3. 2
  4. 4
  5. 4
  6. 6
  7. 6
  8. 9
  9. 14
 10. 15
 11. 16
 12. 16
 13. 17
 14. 18

. end


. input Y1

            Y1
  1. 0
  2. 0
  3. 1
  4. 2
  5. 0
  6. 0
  7. 2
  8. 3
  9. 12
 10. 9
 11. 8
 12. 15
 13. 5
 14. 17

. end


. gen int cluster = (_n+1)/2

. 
. //ssc install tabstatmat  (install the package)
. // save tabstat summary result to matrix
. tabstat Y0, by(cluster) stat(mean) save

Summary for variables: Y0
     by categories of: cluster 

 cluster |      mean
---------+----------
       1 |        .5
       2 |         3
       3 |         5
       4 |       7.5
       5 |      14.5
       6 |        16
       7 |      17.5
---------+----------
   Total |  9.142857
--------------------

. tabstatmat Ybar0, nototal

Ybar0[7,1]
          Y0
1:mean    .5
2:mean     3
3:mean     5
4:mean   7.5
5:mean  14.5
6:mean    16
7:mean  17.5

. mat colnames Ybar0=Ybar0

. 
. tabstat Y1, by(cluster) stat(mean) save

Summary for variables: Y1
     by categories of: cluster 

 cluster |      mean
---------+----------
       1 |         0
       2 |       1.5
       3 |         0
       4 |       2.5
       5 |      10.5
       6 |      11.5
       7 |        11
---------+----------
   Total |  5.285714
--------------------

. tabstatmat Ybar1, nototal

Ybar1[7,1]
          Y1
1:mean     0
2:mean   1.5
3:mean     0
4:mean   2.5
5:mean  10.5
6:mean  11.5
7:mean    11

. mat colnames Ybar1=Ybar1 
. 
. 
. // function to calculate population variance
. cap program drop var_pop

. program define var_pop, rclass
  1.         args varname    
  2.         tempvar x_dev 
  3.         qui sum `varname'
  4.         local avg = r(mean)
  5.         local length = r(N)     
  6.         gen `x_dev' = (`varname'-`avg')^2/`length'
  7.         qui tabstat `x_dev', stat(sum) save
  8.         return scalar variance_pop = el(r(StatTotal),1,1)
  9. end
. 
. 
. // function to calculate population covariance
. cap program drop cor_pop

. program define cor_pop, rclass
  1.         args x y        
  2.         tempvar xy_dev 
  3.         qui sum `x'
  4.         local avg_x = r(mean)
  5.         local length = r(N)     
  6.         
.         qui sum `y'
  7.         local avg_y = r(mean)
  8.                 
.         gen `xy_dev' = (`x'-`avg_x')*(`y'-`avg_y')
  9.         qui tabstat `xy_dev', stat(sum) save
 10.         return scalar cor_pop = el(r(StatTotal),1,1)/`length'
 11. end

. 
. preserve 

. clear

. set obs 7
number of observations (_N) was 0, now 7

. svmat Ybar0, names(col)
number of observations will be reset to 7
Press any key to continue, or Break to abort
number of observations (_N) was 0, now 7

. svmat Ybar1, names(col)
. 
. // var_Ybar0    
. var_pop Ybar0

. scalar var_Ybar0=r(variance_pop)
. 
. // var_Ybar1 
. var_pop Ybar1

. scalar var_Ybar1=r(variance_pop)
. 
. // cov_Ybar0 
. cor_pop Ybar0 Ybar1

. 
. scalar cov_Ybar0=r(cor_pop)

. 
. scalar se_ate = sqrt((1/6)*((4/3)*var_Ybar0+(3/4)*var_Ybar1+2*cov_Ybar0))

. 
. di %8.6f se_ate
4.706192
. 
. restore
\end{lstlisting}

\subsection*{part(b)}

\begin{lstlisting}[language=stata]
 replace cluster = _n
(7 real changes made)

. replace cluster = 15-cluster if (cluster>7)
(7 real changes made)
. 
.         
. clear matrix

. // Ybar0        
. tabstat Y0, by(cluster) stat(mean) save

Summary for variables: Y0
     by categories of: cluster 

 cluster |      mean
---------+----------
       1 |         9
       2 |         9
       3 |         9
       4 |        10
       5 |       9.5
       6 |        10
       7 |       7.5
---------+----------
   Total |  9.142857
--------------------

. tabstatmat Ybar0, nototal

Ybar0[7,1]
         Y0
1:mean    9
2:mean    9
3:mean    9
4:mean   10
5:mean  9.5
6:mean   10
7:mean  7.5

. mat colnames Ybar0=Ybar0

. 
. // Ybar1
. tabstat Y1, by(cluster) stat(mean) save

Summary for variables: Y1
     by categories of: cluster 

 cluster |      mean
---------+----------
       1 |       8.5
       2 |       2.5
       3 |         8
       4 |         5
       5 |       4.5
       6 |         6
       7 |       2.5
---------+----------
   Total |  5.285714
--------------------

. tabstatmat Ybar1, nototal

Ybar1[7,1]
         Y1
1:mean  8.5
2:mean  2.5
3:mean    8
4:mean    5
5:mean  4.5
6:mean    6
7:mean  2.5

. mat colnames Ybar1=Ybar1
        
.         
. preserve 

. clear

. set obs 7
number of observations (_N) was 0, now 7

. svmat Ybar0, names(col)
number of observations will be reset to 7
Press any key to continue, or Break to abort
number of observations (_N) was 0, now 7

. svmat Ybar1, names(col) 

.         
. // var_Ybar0 <- var.pop(Ybar0)  
. var_pop Ybar0

. scalar var_Ybar0=r(variance_pop)

. 
. // var_Ybar1 <- var.pop(Ybar1)
. var_pop Ybar1

. scalar var_Ybar1=r(variance_pop)

. 
. // cov_Ybar0 <- cov.pop(Ybar0,Ybar1)
. cor_pop Ybar0 Ybar1

. scalar cov_Ybar0=r(cor_pop)

. 
. // se_ate
. scalar se_ate = sqrt((1/6)*((4/3)*var_Ybar0+(3/4)*var_Ybar1+2*cov_Ybar0))

. di %8.7f se_ate
0.9766259
. 
. restore 
\end{lstlisting}

\end{document}
