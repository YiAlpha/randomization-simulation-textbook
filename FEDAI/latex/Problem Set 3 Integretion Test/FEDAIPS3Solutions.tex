% %%%%%%%%%%%%%%%%%%%%%%%%%%%%%%%%%%%%%%%%%%%%%%%%%%%%%%%%%%%%%%%%%%%%%%%%%%%%%%%%%%%%%%%%%%%%
% PROBLEM SET LATEX TEMPLATE FILE
% DEFINE DOCUMENT STYLE, LOAD PACKAGES
\documentclass[11pt,notitlepage]{article}\usepackage[]{graphicx}\usepackage[]{color}
%% maxwidth is the original width if it is less than linewidth
%% otherwise use linewidth (to make sure the graphics do not exceed the margin)
\makeatletter
\def\maxwidth{ %
  \ifdim\Gin@nat@width>\linewidth
    \linewidth
  \else
    \Gin@nat@width
  \fi
}
\makeatother

\definecolor{fgcolor}{rgb}{0.345, 0.345, 0.345}
\newcommand{\hlnum}[1]{\textcolor[rgb]{0.686,0.059,0.569}{#1}}%
\newcommand{\hlstr}[1]{\textcolor[rgb]{0.192,0.494,0.8}{#1}}%
\newcommand{\hlcom}[1]{\textcolor[rgb]{0.678,0.584,0.686}{\textit{#1}}}%
\newcommand{\hlopt}[1]{\textcolor[rgb]{0,0,0}{#1}}%
\newcommand{\hlstd}[1]{\textcolor[rgb]{0.345,0.345,0.345}{#1}}%
\newcommand{\hlkwa}[1]{\textcolor[rgb]{0.161,0.373,0.58}{\textbf{#1}}}%
\newcommand{\hlkwb}[1]{\textcolor[rgb]{0.69,0.353,0.396}{#1}}%
\newcommand{\hlkwc}[1]{\textcolor[rgb]{0.333,0.667,0.333}{#1}}%
\newcommand{\hlkwd}[1]{\textcolor[rgb]{0.737,0.353,0.396}{\textbf{#1}}}%

\usepackage{framed}
\makeatletter
\newenvironment{kframe}{%
 \def\at@end@of@kframe{}%
 \ifinner\ifhmode%
  \def\at@end@of@kframe{\end{minipage}}%
  \begin{minipage}{\columnwidth}%
 \fi\fi%
 \def\FrameCommand##1{\hskip\@totalleftmargin \hskip-\fboxsep
 \colorbox{shadecolor}{##1}\hskip-\fboxsep
     % There is no \\@totalrightmargin, so:
     \hskip-\linewidth \hskip-\@totalleftmargin \hskip\columnwidth}%
 \MakeFramed {\advance\hsize-\width
   \@totalleftmargin\z@ \linewidth\hsize
   \@setminipage}}%
 {\par\unskip\endMakeFramed%
 \at@end@of@kframe}
\makeatother

\definecolor{shadecolor}{rgb}{.97, .97, .97}
\definecolor{messagecolor}{rgb}{0, 0, 0}
\definecolor{warningcolor}{rgb}{1, 0, 1}
\definecolor{errorcolor}{rgb}{1, 0, 0}
\newenvironment{knitrout}{}{} % an empty environment to be redefined in TeX


\usepackage{alltt}    % ADD COMMENTS USING A PERCENT SIGN
\usepackage{amsfonts}
\usepackage{amsthm}
\usepackage{amsmath, booktabs}
\usepackage{mathtools}
\usepackage{amssymb}
\usepackage{subfig}
\usepackage{setspace}
\usepackage{fullpage}
\usepackage{verbatim}
\usepackage{graphicx}
\usepackage{tabularx}
\usepackage{longtable}
\usepackage{multicol}
\usepackage{multirow}
\setlength{\parindent}{0in}		% uncomment to remove indent at start of paragraphs
\usepackage{pdflscape}
\usepackage[english]{babel}
\usepackage[pdftex]{hyperref}
\usepackage{natbib}
\usepackage{caption}
\usepackage{amsmath}
\usepackage{amsfonts}
\usepackage{graphics}
\usepackage{multirow}
\usepackage{graphics}
\usepackage{hyperref}
\usepackage{longtable}
\usepackage{latexsym}
\usepackage{rotating}
\usepackage{setspace}
\usepackage{layouts} 
\usepackage[titletoc]{appendix}
\DeclareGraphicsExtensions{.pdf,.jpg,.png}
\usepackage[margin=1in]{geometry}
\usepackage{enumerate}
\usepackage{float}

\usepackage{xcolor}
\usepackage[printwatermark]{xwatermark}
\newwatermark[allpages,color=black!50,angle=45,scale=1,xpos=0,ypos=0]{DO NOT DISTRIBUTE}

% ---------------------------------------------------------------------
% Syntax Rvised from
% - https://github.com/isagalaev/highlight.js/blob/master/src/languages/stata.js
% - https://github.com/jpitblado/vim-stata/blob/master/syntax/stata.vim
% - http://fmwww.bc.edu/RePEc/bocode/s/synlightlist.ado

\usepackage{listings}



% ---------------------------------------------------------------------
% Stata language definition

\lstdefinelanguage{stata}{
  sensitive=true,
  %
  % Macros, global and local
  alsoletter={\{\}0123456789},
  keywordsprefix=\$,
  morecomment=[n][keywordstyle9]{`}{'},
  morekeywords={},
  %
  % Comments
  morecomment=[f][\color{Green}\slshape][0]*,
  morecomment=[l]{//},
  morecomment=[s]{/*}{*/},
  %
  % Strings
  morecomment=[n][\color{Maroon}]{`"}{"'},
  morestring=[b]",
  %
  % Add-ons and system Commands
  morekeywords=[2]{
    if ,else ,in ,foreach ,forv ,forva ,forval ,forvalu ,forvalue
    ,forvalues ,by ,bys ,bysort ,xi ,quietly ,qui ,capture ,about
    ,ac ,ac_7 ,acprplot ,acprplot_7 adjust ,ado ,adopath ,adoupdate
    ,alpha ,ameans ,an ,ano ,anov ,anova ,anova_estat ,anova_terms
    ,anovadef ,aorder ,ap ,app ,appe ,appen ,append ,arch ,arch_dr
    ,arch_estat ,arch_p ,archlm ,areg ,areg_p ,args ,arima ,arima_dr
    ,arima_estat ,arima_p ,as ,asmprobit ,asmprobit_estat ,asmprobit_lf
    ,asmprobit_mfx__dlg ,asmprobit_p ,ass ,asse ,asser ,assert ,avplot
    ,avplot_7 ,avplots ,avplots_7 bcskew0 ,bgodfrey ,binreg ,bip0_lf
    ,biplot ,bipp_lf ,bipr_lf ,bipr_p ,biprobit ,bitest ,bitesti
    ,bitowt ,blogit ,bmemsize ,boot ,bootsamp ,bootstrap ,bootstrap_8
    ,boxco_l ,boxco_p ,boxcox ,boxcox_6 ,boxcox_p ,bprobit ,br ,break
    ,brier ,bro ,brow ,brows ,browse ,brr ,brrstat ,bs ,bs_7 ,bsampl_w
    ,bsample ,bsample_7 ,bsqreg ,bstat ,bstat_7 ,bstat_8 ,bstrap
    ,bstrap_7 ,ca ,ca_estat ,ca_p ,cabiplot ,camat ,canon ,canon_8
    ,canon_8_p ,canon_estat ,canon_p ,cap ,caprojection ,capt ,captu
    ,captur ,capture ,cat ,cc ,cchart ,cchart_7 ,cci ,cd ,censobs_table
    ,centile ,cf ,char ,chdir ,checkdlgfiles ,checkestimationsample
    ,checkhlpfiles ,checksum ,chelp ,ci ,cii ,cl ,class ,classutil
    ,clear ,cli ,clis ,clist ,clo ,clog ,clog_lf ,clog_p ,clogi
    ,clogi_sw ,clogit ,clogit_lf ,clogit_p ,clogitp ,clogl_sw ,cloglog
    ,clonevar ,clslistarray ,cluster ,cluster_measures ,cluster_stop
    ,cluster_tree ,cluster_tree_8 ,clustermat ,cmdlog ,cnr ,cnre
    ,cnreg ,cnreg_p ,cnreg_sw ,cnsreg ,codebook ,collaps4 ,collapse
    ,colormult_nb ,colormult_nw ,compare ,compress ,conf ,confi
    ,confir ,confirm ,conren ,cons ,const ,constr ,constra ,constrai
    ,constrain ,constraint ,continue ,contract ,copy ,copyright
    ,copysource ,cor ,corc ,corr ,corr2data ,corr_anti ,corr_kmo
    ,corr_smc ,corre ,correl ,correla ,correlat ,correlate ,corrgram
    ,cou ,coun ,count ,cox ,cox_p ,cox_sw ,coxbase ,coxhaz ,coxvar
    ,cprplot ,cprplot_7 ,crc ,cret ,cretu ,cretur ,creturn ,cross ,cs
    ,cscript ,cscript_log ,csi ,ct ,ct_is ,ctset ,ctst_5 ,ctst_st
    ,cttost ,cumsp ,cumsp_7 ,cumul ,cusum ,cusum_7 ,cutil ,d ,datasig
    ,datasign ,datasigna ,datasignat ,datasignatu ,datasignatur
    ,datasignature ,datetof ,db ,dbeta ,de ,dec ,deco ,decod ,decode
    ,deff ,delim ,des ,desc ,descr ,descri ,describ ,describe ,destring
    ,dfbeta ,dfgls ,dfuller ,di ,di_g ,dir ,dirstats ,dis ,discard
    ,disp ,disp_res ,disp_s ,displ ,displa ,display ,distinct ,do
    ,doe ,doed ,doedi ,doedit ,dotplot ,dotplot_7 ,dprobit ,drawnorm
    ,drop ,ds ,ds_util ,dstdize ,duplicates ,durbina ,dwstat ,dydx ,e
    ,ed ,end ,edi ,edit ,egen ,eivreg ,emdef ,en ,enc ,enco ,encod ,encode
    ,eq ,erase ,ereg ,ereg_lf ,ereg_p ,ereg_sw ,ereghet ,ereghet_glf
    ,ereghet_glf_sh ,ereghet_gp ,ereghet_ilf ,ereghet_ilf_sh ,ereghet_ip
    ,eret ,eretu ,eretur ,ereturn ,err ,erro ,error ,est ,est_cfexist
    ,est_cfname ,est_clickable ,est_expand ,est_hold ,est_table
    ,est_unhold ,est_unholdok ,estat ,estat_default ,estat_summ
    ,estat_vce_only ,esti ,estimates ,etodow ,etof ,etomdy ,ex ,exi
    ,exit ,expand ,expandcl ,fac ,fact ,facto ,factor ,factor_estat
    ,factor_p ,factor_pca_rotated ,factor_rotate ,factormat ,fcast
    ,fcast_compute ,fcast_graph ,fdades ,fdadesc ,fdadescr ,fdadescri
    ,fdadescrib ,fdadescribe ,fdasav ,fdasave ,fdause ,fh_st
    ,open ,read ,close ,filefilter ,fillin
    ,find_hlp_file ,findfile ,findit ,findit_7 ,fit ,fl ,fli ,flis
    ,flist ,for5_0 ,form ,forma ,format ,fpredict ,frac_154 ,frac_adj
    ,frac_chk ,frac_cox ,frac_ddp ,frac_dis ,frac_dv ,frac_in ,frac_mun
    ,frac_pp ,frac_pq ,frac_pv ,frac_wgt ,frac_xo ,fracgen ,fracplot
    ,fracplot_7 ,fracpoly ,fracpred ,fron_ex ,fron_hn ,fron_p ,fron_tn
    ,fron_tn2 ,frontier ,ftodate ,ftoe ,ftomdy ,ftowdate ,g ,gamhet_glf
    ,gamhet_gp ,gamhet_ilf ,gamhet_ip ,gamma ,gamma_d2 ,gamma_p
    ,gamma_sw ,gammahet ,gdi_hexagon ,gdi_spokes ,ge ,gen ,gene ,gener
    ,genera ,generat ,generate ,genrank ,genstd ,genvmean ,gettoken
    ,gl ,gladder ,gladder_7 ,glim_l01 ,glim_l02 glim_l03 ,glim_l04
    ,glim_l05 ,glim_l06 ,glim_l07 ,glim_l08 ,glim_l09 ,glim_l10 glim_l11
    ,glim_l12 ,glim_lf ,glim_mu ,glim_nw1 ,glim_nw2 ,glim_nw3 ,glim_p
    ,glim_v1 ,glim_v2 ,glim_v3 ,glim_v4 ,glim_v5 ,glim_v6 ,glim_v7 ,glm
    ,glm_6 glm_p ,glm_sw ,glmpred ,glo ,glob ,globa ,global ,glogit
    ,glogit_8 ,glogit_p ,gmeans ,gnbre_lf ,gnbreg ,gnbreg_5 ,gnbreg_p
    ,gomp_lf ,gompe_sw ,gomper_p ,gompertz ,gompertzhet ,gomphet_glf
    ,gomphet_glf_sh ,gomphet_gp ,gomphet_ilf ,gomphet_ilf_sh ,gomphet_ip
    ,gphdot ,gphpen ,gphprint ,gprefs ,gprobi_p ,gprobit ,gprobit_8
    ,gr ,gr7 ,gr_copy ,gr_current ,gr_db ,gr_describe ,gr_dir ,gr_draw
    ,gr_draw_replay ,gr_drop ,gr_edit ,gr_editviewopts ,gr_example
    ,gr_example2 gr_export ,gr_print ,gr_qscheme ,gr_query ,gr_read
    ,gr_rename ,gr_replay ,gr_save ,gr_set ,gr_setscheme ,gr_table
    ,gr_undo ,gr_use ,graph ,graph7 grebar ,greigen ,greigen_7
    ,greigen_8 ,grmeanby ,grmeanby_7 ,gs_fileinfo ,gs_filetype
    ,gs_graphinfo ,gs_stat ,gsort ,gwood ,h ,hadimvo ,hareg ,hausman
    ,haver ,he ,heck_d2 ,heckma_p ,heckman ,heckp_lf ,heckpr_p ,heckprob
    ,hel ,help ,hereg ,hetpr_lf ,hetpr_p ,hetprob ,hettest ,hexdump
    ,hilite ,hist ,hist_7 histogram ,hlogit ,hlu ,hmeans ,hotel
    ,hotelling ,hprobit ,hreg ,hsearch ,icd9 ,icd9_ff ,icd9p ,iis
    ,impute ,imtest ,import ,inbase ,include ,inf ,infi ,infil ,infile ,infix
    ,inp ,inpu ,input ,ins ,insheet ,insp ,inspe ,inspec ,inspect ,integ
    ,inten ,intreg ,intreg_7 ,intreg_p ,intrg2_ll ,intrg_ll ,intrg_ll2
    ,ipolate ,iqreg ,ir ,irf ,irf_create ,irfm ,iri ,is_svy ,is_svysum
    ,isid ,istdize ,ivprob_1_lf ,ivprob_lf ,ivprobit ,ivprobit_p ,ivreg
    ,ivreg_footnote ,ivtob_1_lf ,ivtob_lf ,ivtobit ,ivtobit_p ,jackknife
    ,jacknife ,jknife ,jknife_6 ,jknife_8 ,jkstat ,joinby ,kalarma1
    ,kap ,kap_3 ,kapmeier ,kappa ,kapwgt ,kdensity ,kdensity_7 keep
    ,ksm ,ksmirnov ,ktau ,kwallis ,l ,la ,lab ,labe ,label ,labelbook
    ,ladder ,levels ,levelsof ,leverage ,lfit ,lfit_p ,li ,lincom ,line
    ,linktest ,lis ,list ,lloghet_glf ,lloghet_glf_sh ,lloghet_gp
    ,lloghet_ilf ,lloghet_ilf_sh ,lloghet_ip ,llogi_sw ,llogis_p
    ,llogist ,llogistic ,llogistichet ,lnorm_lf ,lnorm_sw ,lnorma_p
    ,lnormal ,lnormalhet ,lnormhet_glf ,lnormhet_glf_sh ,lnormhet_gp
    ,lnormhet_ilf ,lnormhet_ilf_sh ,lnormhet_ip ,lnskew0 ,loadingplot
    ,loc ,loca ,local ,logi ,logis_lf ,logistic ,logistic_p
    ,logit ,logit_estat ,logit_p ,loglogs ,logrank ,loneway ,lookfor
    ,lookup ,lowess ,lowess_7 ,lpredict ,lrecomp ,lroc ,lroc_7 ,lrtest
    ,ls ,lsens ,lsens_7 ,lsens_x ,lstat ,ltable ,ltable_7 ,ltriang
    ,lv ,lvr2plot ,lvr2plot_7 ,m ,ma ,mac ,macr ,macro ,makecns ,man
    ,manova ,manova_estat ,manova_p ,manovatest ,mantel ,mark ,markin
    ,markout ,marksample ,mat ,mat_capp ,mat_order ,mat_put_rr ,mat_rapp
    ,mata ,mata_clear ,mata_describe ,mata_drop ,mata_matdescribe
    ,mata_matsave ,mata_matuse ,mata_memory ,mata_mlib ,mata_mosave
    ,mata_rename ,mata_which ,matalabel ,matcproc ,matlist ,matname
    ,matr ,matri ,matrix ,matrix_input__dlg ,matstrik ,mcc ,mcci ,md0_
    ,md1_ ,md1debug_ ,md2_ ,md2debug_ ,mds ,mds_estat ,mds_p ,mdsconfig
    ,mdslong ,mdsmat ,mdsshepard ,mdytoe ,mdytof ,me_derd ,mean ,means
    ,median ,memory ,memsize ,meqparse ,mer ,merg ,merge ,mfp ,mfx
    ,mhelp ,mhodds ,minbound ,mixed_ll ,mixed_ll_reparm ,mkassert
    ,mkdir ,mkmat ,mkspline ,ml ,ml_5 ml_adjs ,ml_bhhhs ,ml_c_d
    ,ml_check ,ml_clear ,ml_cnt ,ml_debug ,ml_defd ,ml_e0 ml_e0_bfgs
    ,ml_e0_cycle ,ml_e0_dfp ,ml_e0i ,ml_e1 ,ml_e1_bfgs ,ml_e1_bhhh
    ,ml_e1_cycle ,ml_e1_dfp ,ml_e2 ,ml_e2_cycle ,ml_ebfg0 ,ml_ebfr0
    ,ml_ebfr1 ml_ebh0q ,ml_ebhh0 ,ml_ebhr0 ,ml_ebr0i ,ml_ecr0i ,ml_edfp0
    ,ml_edfr0 ,ml_edfr1 ml_edr0i ,ml_eds ,ml_eer0i ,ml_egr0i ,ml_elf
    ,ml_elf_bfgs ,ml_elf_bhhh ,ml_elf_cycle ,ml_elf_dfp ,ml_elfi
    ,ml_elfs ,ml_enr0i ,ml_enrr0 ,ml_erdu0 ml_erdu0_bfgs ,ml_erdu0_bhhh
    ,ml_erdu0_bhhhq ,ml_erdu0_cycle ,ml_erdu0_dfp ,ml_erdu0_nrbfgs
    ,ml_exde ,ml_footnote ,ml_geqnr ,ml_grad0 ,ml_graph ,ml_hbhhh
    ,ml_hd0 ,ml_hold ,ml_init ,ml_inv ,ml_log ,ml_max ,ml_mlout
    ,ml_mlout_8 ,ml_model ,ml_nb0 ,ml_opt ,ml_p ,ml_plot ,ml_query
    ,ml_rdgrd ,ml_repor ,ml_s_e ,ml_score ,ml_searc ,ml_technique
    ,ml_unhold ,mleval ,mlf_ ,mlmatbysum ,mlmatsum ,mlog ,mlogi ,mlogit
    ,mlogit_footnote ,mlogit_p ,mlopts ,mlsum ,mlvecsum ,mnl0_ ,mor
    ,more ,mov ,move ,mprobit ,mprobit_lf ,mprobit_p ,mrdu0_ ,mrdu1_
    ,mvdecode ,mvencode ,mvreg ,mvreg_estat ,nbreg ,nbreg_al
    ,nbreg_lf ,nbreg_p ,nbreg_sw ,nestreg ,net ,newey ,newey_7 ,newey_p
    ,news ,nl ,nl_7 ,nl_9 ,nl_9_p ,nl_p ,nl_p_7 nlcom ,nlcom_p ,nlexp2
    ,nlexp2_7 ,nlexp2a ,nlexp2a_7 ,nlexp3 ,nlexp3_7 ,nlgom3 nlgom3_7
    ,nlgom4 ,nlgom4_7 ,nlinit ,nllog3 ,nllog3_7 ,nllog4 ,nllog4_7
    ,nlog_rd ,nlogit ,nlogit_p ,nlogitgen ,nlogittree ,nlpred ,no
    ,nobreak ,noi ,nois ,noisi ,noisil ,noisily ,note ,notes ,notes_dlg
    ,nptrend ,numlabel ,numlist ,odbc ,old_ver ,olo ,olog ,ologi
    ,ologi_sw ,ologit ,ologit_p ,ologitp ,on ,onew ,onewa ,oneway
    ,op_colnm ,op_comp ,op_diff ,op_inv ,op_str ,opr ,opro ,oprob
    ,oprob_sw ,oprobi ,oprobi_p ,oprobit ,oprobitp ,opts_exclusive
    ,order ,orthog ,orthpoly ,ou ,out ,outf ,outfi ,outfil ,outfile
    ,outs ,outsh ,outshe ,outshee ,outsheet ,ovtest ,pac ,pac_7 ,palette
    ,parse ,parse_dissim ,pause ,pca ,pca_8 pca_display ,pca_estat
    ,pca_p ,pca_rotate ,pcamat ,pchart ,pchart_7 ,pchi ,pchi_7 ,pcorr
    ,pctile ,pentium ,pergram ,pergram_7 ,permute ,permute_8 ,personal
    ,peto_st ,pkcollapse ,pkcross ,pkequiv ,pkexamine ,pkexamine_7
    ,pkshape ,pksumm ,pksumm_7 ,pl ,plo ,plot ,plugin ,pnorm ,pnorm_7
    ,poisgof ,poiss_lf ,poiss_sw ,poisso_p ,poisson ,poisson_estat
    ,post ,postclose ,postfile ,postutil ,pperron ,pr ,prais ,prais_e
    ,prais_e2 ,prais_p ,predict ,predictnl ,preserve ,print ,pro ,prob
    ,probi ,probit ,probit_estat ,probit_p ,proc_time ,procoverlay
    ,procrustes ,procrustes_estat ,procrustes_p ,profiler ,prog ,progr
    ,progra ,program ,prop ,proportion ,prtest ,prtesti ,pwcorr ,pwd
    ,q ,s ,qby ,qbys ,qchi ,qchi_7 ,qladder ,qladder_7 ,qnorm ,qnorm_7
    ,qqplot ,qqplot_7 ,qreg ,qreg_c ,qreg_p ,qreg_sw ,qu ,quadchk
    ,quantile ,quantile_7 ,que ,quer ,query ,range ,ranksum ,ratio
    ,rchart ,rchart_7 ,rcof ,recast ,reclink ,recode ,reg ,reg3
    ,reg3_p ,regdw ,regr ,regre ,regre_p2 ,regres ,regres_p ,regress
    ,regress_estat ,regriv_p ,remap ,ren ,rena ,renam ,rename ,renpfix
    ,repeat ,replace ,report ,reshape ,restore ,ret ,retu ,retur ,return
    ,rm ,rmdir ,robvar ,roccomp ,roccomp_7 ,roccomp_8 ,rocf_lf ,rocfit
    ,rocfit_8 ,rocgold ,rocplot ,rocplot_7 ,roctab ,roctab_7 ,rolling
    ,rologit ,rologit_p ,rot ,rota ,rotat ,rotate ,rotatemat ,rclass ,rreg
    ,rreg_p ,ru ,run ,runtest ,rvfplot ,rvfplot_7 ,rvpplot ,rvpplot_7
    ,sa ,safesum ,sampsi ,sav ,save ,savedresults ,saveold ,sc
    ,sca ,scal ,scala ,scalar ,scatter ,scm_mine ,sco ,scob_lf ,scob_p
    ,scobi_sw ,scobit ,scor ,score ,scoreplot ,scoreplot_help ,scree
    ,screeplot ,screeplot_help ,sdtest ,sdtesti ,se ,seed ,search ,separate
    ,seperate ,serrbar ,serrbar_7 ,serset ,set ,set_defaults ,sfrancia
    ,sh ,she ,shel ,shell ,shewhart ,shewhart_7 ,signestimationsample
    ,signrank ,signtest ,simul ,simul_7 simulate ,simulate_8 ,sktest
    ,sleep ,slogit ,slogit_d2 ,slogit_p ,smooth ,snapspan ,so ,sor
    ,sort ,spearman ,spikeplot ,spikeplot_7 ,spikeplt ,spline_x ,split
    ,sqreg ,sqreg_p ,sret ,sretu ,sretur ,sreturn ,ssc ,st ,st_ct ,st_hc
    ,st_hcd ,st_hcd_sh ,st_is ,st_issys ,st_note ,st_promo ,st_set
    ,st_show ,st_smpl ,st_subid ,stack ,stata ,statsby ,statsby_8 ,stbase
    ,stci ,stci_7 ,stcox ,stcox_estat ,stcox_fr ,stcox_fr_ll ,stcox_p
    ,stcox_sw ,stcoxkm ,stcoxkm_7 ,stcstat ,stcurv ,stcurve ,stcurve_7
    ,stdes ,stem ,stepwise ,stereg ,stfill ,stgen ,stir ,stjoin ,stmc
    ,stmh ,stphplot ,stphplot_7 ,stphtest ,stphtest_7 ,stptime ,strate
    ,strate_7 ,streg ,streg_sw ,streset ,sts ,sts_7 ,stset ,stsplit
    ,stsum ,sttocc ,sttoct ,stvary ,stweib ,su ,suest ,suest_8 ,sum
    ,summ ,summa ,summar ,summari ,summariz ,summarize ,sunflower
    ,sureg ,survcurv ,survsum ,svar ,svar_p ,svmat ,svy ,svy_disp
    ,svy_dreg ,svy_est ,svy_est_7 ,svy_estat ,svy_get ,svy_gnbreg_p
    ,svy_head ,svy_header ,svy_heckman_p ,svy_heckprob_p ,svy_intreg_p
    ,svy_ivreg_p ,svy_logistic_p ,svy_logit_p ,svy_mlogit_p ,svy_nbreg_p
    ,svy_ologit_p ,svy_oprobit_p ,svy_poisson_p ,svy_probit_p
    ,svy_regress_p ,svy_sub ,svy_sub_7 ,svy_x ,svy_x_7 ,svy_x_p ,svydes
    ,svydes_8 ,svygen ,svygnbreg ,svyheckman ,svyheckprob ,svyintreg
    ,svyintreg_7 ,svyintrg ,svyivreg ,svylc ,svylog_p ,svylogit
    ,svymarkout ,svymarkout_8 ,svymean ,svymlog ,svymlogit ,svynbreg
    ,svyolog ,svyologit ,svyoprob ,svyoprobit ,svyopts ,svypois
    ,svypois_7 svypoisson ,svyprobit ,svyprobt ,svyprop ,svyprop_7
    ,svyratio ,svyreg ,svyreg_p ,svyregress ,svyset ,svyset_7 ,svyset_8
    ,svytab ,svytab_7 ,svytest ,svytotal ,sw ,sw_8 ,swcnreg ,swcox
    ,swereg ,swilk ,swlogis ,swlogit ,swologit ,swoprbt ,swpois
    ,swprobit ,swqreg ,swtobit ,swweib ,symmetry ,symmi ,symplot
    ,symplot_7 syntax ,sysdescribe ,sysdir ,sysuse ,szroeter ,ta ,tab
    ,tab1 ,tab2 ,tab_or ,tabd ,tabdi ,tabdis ,tabdisp ,tabi ,table
    ,tabodds ,tabodds_7 ,tabstat ,,tabstatmat ,tabu ,tabul ,tabula ,tabulat ,tabulate
    ,te ,tempfile ,tempname ,tempvar ,tes ,tsrtest ,ritest ,testnl ,testparm
    ,teststd ,tetrachoric ,time_it ,timer ,tis ,tob ,tobi ,tobit
    ,tobit_p ,tobit_sw ,token ,tokeni ,tokeniz ,tokenize ,tostring
    ,total ,translate ,translator ,transmap ,treat_ll ,treatr_p
    ,treatreg ,trim ,trnb_cons ,trnb_mean ,trpoiss_d2 ,trunc_ll
    ,truncr_p ,truncreg ,tsappend ,tset ,tsfill ,tsline ,tsline_ex
    ,tsreport ,tsrevar ,tsrline ,tsset ,tssmooth ,tsunab ,ttest
    ,ttesti ,tut_chk ,tut_wait ,tutorial ,tw ,tware_st ,twoway
    ,twoway__fpfit_serset ,twoway__function_gen ,twoway__histogram_gen
    ,twoway__ipoint_serset ,twoway__ipoints_serset ,twoway__kdensity_gen
    ,twoway__lfit_serset ,twoway__normgen_gen ,twoway__pci_serset
    ,twoway__qfit_serset ,twoway__scatteri_serset ,twoway__sunflower_gen
    ,twoway_ksm_serset ,ty ,typ ,type ,typeof ,u ,unab ,unabbrev
    ,unabcmd ,update ,us ,use ,uselabel ,var ,var_mkcompanion
    ,var_p ,varbasic ,varfcast ,vargranger ,varirf ,varirf_add
    ,varirf_cgraph ,varirf_create ,varirf_ctable ,varirf_describe
    ,varirf_dir ,varirf_drop ,varirf_erase ,varirf_graph ,varirf_ograph
    ,varirf_rename ,varirf_set ,varirf_table ,varlist ,varlmar
    ,varnorm ,varsoc ,varstable ,varstable_w ,varstable_w2 ,varwle
    ,vce ,vec ,vec_fevd ,vec_mkphi ,vec_p ,vec_p_w ,vecirf_create
    ,veclmar ,veclmar_w ,vecnorm ,vecnorm_w ,vecrank ,vecstable
    ,verinst ,vers ,versi ,versio ,version ,view ,viewsource ,vif
    ,vwls ,wdatetof ,webdescribe ,webseek ,webuse ,weib1_lf ,weib2_lf
    ,weib_lf ,weib_lf0 weibhet_glf ,weibhet_glf_sh ,weibhet_glfa
    ,weibhet_glfa_sh ,weibhet_gp ,weibhet_ilf ,weibhet_ilf_sh
    ,weibhet_ilfa ,weibhet_ilfa_sh ,weibhet_ip ,weibu_sw ,weibul_p
    ,weibull ,weibull_c ,weibull_s ,weibullhet ,wh ,whelp ,whi ,which
    ,whil ,while ,wilc_st ,wilcoxon ,win ,wind ,windo ,window ,winexec
    ,wntestb ,wntestb_7 ,wntestq ,xchart ,xchart_7 ,xcorr ,xcorr_7 ,xi
    ,xi_6 ,xmlsav ,xmlsave ,xmluse ,xpose ,xsh ,xshe ,xshel ,xshell
    ,xt_iis ,xt_tis ,xtab_p ,xtabond ,xtbin_p ,xtclog ,xtcloglog
    ,xtcloglog_8 ,xtcloglog_d2 ,xtcloglog_pa_p ,xtcloglog_re_p ,xtcnt_p
    ,xtcorr ,xtdata ,xtdes ,xtfront_p ,xtfrontier ,xtgee ,xtgee_elink
    ,xtgee_estat ,xtgee_makeivar ,xtgee_p ,xtgee_plink ,xtgls ,xtgls_p
    ,xthaus ,xthausman ,xtht_p ,xthtaylor ,xtile ,xtint_p ,xtintreg
    ,xtintreg_8 ,xtintreg_d2 xtintreg_p ,xtivp_1 ,xtivp_2 ,xtivreg
    ,xtline ,xtline_ex ,xtlogit ,xtlogit_8 xtlogit_d2 ,xtlogit_fe_p
    ,xtlogit_pa_p ,xtlogit_re_p ,xtmixed ,xtmixed_estat ,xtmixed_p
    ,xtnb_fe ,xtnb_lf ,xtnbreg ,xtnbreg_pa_p ,xtnbreg_refe_p ,xtpcse
    ,xtpcse_p ,xtpois ,xtpoisson ,xtpoisson_d2 ,xtpoisson_pa_p
    ,xtpoisson_refe_p ,xtpred ,xtprobit ,xtprobit_8 ,xtprobit_d2
    ,xtprobit_re_p ,xtps_fe ,xtps_lf ,xtps_ren ,xtps_ren_8 ,xtrar_p
    ,xtrc ,xtrc_p ,xtrchh ,xtrefe_p ,xtreg ,xtreg_be ,xtreg_fe
    ,xtreg_ml ,xtreg_pa_p ,xtreg_re ,xtregar ,xtrere_p ,xtset
    ,xtsf_ll ,xtsf_llti ,xtsum ,xttab ,xttest0 ,xttobit ,xttobit_8
    ,xttobit_p ,xttrans ,yx ,yxview__barlike_draw ,yxview_area_draw
    ,yxview_bar_draw ,yxview_dot_draw ,yxview_dropline_draw
    ,yxview_function_draw ,yxview_iarrow_draw ,yxview_ilabels_draw
    ,yxview_normal_draw ,yxview_pcarrow_draw ,yxview_pcbarrow_draw
    ,yxview_pccapsym_draw ,yxview_pcscatter_draw ,yxview_pcspike_draw
    ,yxview_rarea_draw ,yxview_rbar_draw ,yxview_rbarm_draw
    ,yxview_rcap_draw ,yxview_rcapsym_draw ,yxview_rconnected_draw
    ,yxview_rline_draw ,yxview_rscatter_draw ,yxview_rspike_draw
    ,yxview_spike_draw ,yxview_sunflower_draw ,zap_s ,zinb ,zinb_llf
    ,zinb_plf ,zip ,zip_llf ,zip_p ,zip_plf ,zt_ct_5 ,zt_hc_5 ,zt_hcd_5
    ,zt_is_5 ,zt_iss_5 ,zt_sho_5 zt_smp_5 ,ztbase_5 ,ztcox_5 ,ztdes_5
    ,ztereg_5 ,ztfill_5 ,ztgen_5 ,ztir_5 ztjoin_5 ,ztnb ,ztnb_p ,ztp
    ,ztp_p ,zts_5 ,ztset_5 ,ztspli_5 ,ztsum_5 ,zttoct_5 ztvary_5
    ,ztweib_5
  },
  %
  % Built-in functions
  morekeywords=[3]{
    Cdhms ,Chms ,Clock ,Cmdyhms ,Cofc ,Cofd ,F ,Fden ,Ftail ,I ,J
    ,_caller ,abbrev ,abs ,acos ,acosh ,asin ,asinh ,atan ,atan2
    ,atanh ,autocode ,betaden ,binomial ,binomialp ,binomialtail
    ,binormal ,bofd ,byteorder ,ceil ,char ,chi2 ,chi2den ,chi2tail
    ,cholesky ,chop ,clip ,clock ,cloglog ,cofC ,cofd ,colnumb ,colsof
    ,comb ,cond ,corr ,cos ,cosh ,d ,daily ,date ,day ,det ,dgammapda
    ,dgammapdada ,dgammapdadx ,dgammapdx ,dgammapdxdx ,dhms ,diag
    ,diag0cnt ,digamma ,dofC ,dofb ,dofc ,dofh ,dofm ,dofq ,dofw ,dofy
    ,dow ,doy ,dunnettprob ,e ,el ,epsdouble ,epsfloat ,exp ,fileexists
    ,fileread ,filereaderror ,filewrite ,float ,floor ,fmtwidth
    ,gammaden ,gammap ,gammaptail ,get ,group ,h ,hadamard ,halfyear
    ,halfyearly ,has_eprop ,hh ,hhC ,hms ,hofd ,hours ,hypergeometric
    ,hypergeometricp ,ibeta ,ibetatail ,index ,indexnot ,inlist
    ,inrange ,int ,inv ,invF ,invFtail ,invbinomial ,invbinomialtail
    ,invchi2 ,invchi2tail ,invcloglog ,invdunnettprob ,invgammap
    ,invgammaptail ,invibeta ,invibetatail ,invlogit ,invnFtail
    ,invnbinomial ,invnbinomialtail ,invnchi2 ,invnchi2tail ,invnibeta
    ,invnorm ,invnormal ,invnttail ,invpoisson ,invpoissontail ,invsym
    ,invt ,invttail ,invtukeyprob ,irecode ,issym ,issymmetric ,itrim
    ,length ,ln ,lnfact ,lnfactorial ,lngamma ,lnnormal ,lnnormalden
    ,log10 ,logit ,lower ,ltrim ,m ,match ,matmissing ,matrix
    ,matuniform ,max ,maxbyte ,maxdouble ,maxfloat ,maxint ,maxlong ,mdy
    ,mdyhms ,mi ,mi ,min ,minbyte ,mindouble ,minfloat ,minint ,minlong
    ,minutes ,mm ,mmC ,mod ,mofd ,month ,monthly ,mreldif
    ,msofhours ,msofminutes ,msofseconds ,nF ,nFden ,nFtail ,nbetaden
    ,nbinomial ,nbinomialp ,nbinomialtail ,nchi2 ,nchi2den ,nchi2tail
    ,nibeta ,norm ,normal ,normalden ,normd ,npnF ,npnchi2 ,npnt ,nt
    ,ntden ,nttail ,nullmat ,plural ,poisson ,poissonp ,poissontail
    ,proper ,q ,qofd ,quarter ,quarterly ,r ,rbeta ,rbinomial ,rchi2
    real ,recode ,regexm ,regexr ,regexs ,reldif ,replay ,return
    ,reverse ,rgamma ,rhypergeometric ,rnbinomial ,rnormal ,round
    ,rownumb ,rowsof ,rpoisson ,rt ,rtrim ,runiform ,s ,scalar ,seconds
    ,sign ,sin ,sinh ,smallestdouble ,soundex ,soundex_nara ,sqrt ,ss
    ,ssC ,strcat ,strdup ,string ,strlen ,strlower ,strltrim ,strmatch
    ,strofreal ,strpos ,strproper ,strreverse ,strrtrim ,strtoname
    ,strtrim ,strupper ,subinstr ,subinword ,substr ,sum ,sweep ,syminv
    ,t ,tC ,tan ,tanh ,tc ,td ,tden ,th ,tin ,tm ,tq ,trace ,trigamma
    ,trim ,trunc ,ttail ,tukeyprob ,tw ,twithin ,uniform ,upper ,vec
    ,vecdiag ,w ,week ,weekly ,wofd ,word ,wordcount ,year ,yearly
    ,yh ,ym ,yofd ,yq ,yw
  },
}

% ---------------------------------------------------------------------
% Stata highligh style

\providecommand{\textcolordummy}[2]{#2}
\lstalias{Stata}{stata}






\definecolor{ao(english)}{rgb}{0.0, 0.5, 0.0}
\definecolor{brightmaroon}{rgb}{0.76, 0.13, 0.28}
\definecolor{codepurple}{rgb}{0.70,0.20,0.60}
\definecolor{basiccode}{rgb}{0.18,0.18,0.18}



\lstdefinestyle{mystyle}{
basicstyle= \footnotesize\ttfamily\color{basiccode},
backgroundcolor=\color{shadecolor},
commentstyle=\color{ao(english)},
keywordstyle=\color{brightmaroon},
stringstyle=\color{codepurple},
breakatwhitespace=false,
breaklines=true,
captionpos=b,
keepspaces=false,
numbersep=5pt,
showspaces=false,
showstringspaces=false,
showtabs=false,
tabsize = 2
}

\lstset{style=mystyle}
\newcolumntype{L}[1]{>{\raggedright\let\newline\\\arraybackslash\hspace{0pt}}m{#1}}
\newcolumntype{C}[1]{>{\centering\let\newline\\\arraybackslash\hspace{0pt}}m{#1}}
\newcolumntype{R}[1]{>{\raggedleft\let\newline\\\arraybackslash\hspace{0pt}}m{#1}}
% FONTS
\usepackage[T1]{fontenc}					% always use this no matter what

% uncomment any one of these to see what it does to your font!
%\usepackage{pxfonts}
%\usepackage{cmbright}
%\usepackage{txfonts}
%\usepackage[adobe-utopia]{mathdesign}
%\usepackage{kpfonts}
%\usepackage{lmodern}
%\usepackage{newtxtext,newtxmath}



\IfFileExists{upquote.sty}{\usepackage{upquote}}{}
\begin{document}
\title{Field Experiments: Design, Analysis and Interpretation \\
Solutions for Chapter 3 Exercises}
\author{Alan S. Gerber and Donald P. Green\footnote{Solutions prepared by Peter M. Aronow and revised by Alexander Coppock}}
\date{\today}
\maketitle

\section*{Question 1}
Important concepts: [10 points]
\begin{enumerate}[a)]
\item What is a standard error?  What is the difference between a standard error and a standard deviation?\\
Answer:\\
The standard error is a measure of the statistical uncertainty surrounding a parameter estimate. The standard error is a measure of dispersion in a sampling distribution; the standard deviation is the measure of dispersion of any distribution but is most often used to describe the dispersion in an observed variable. The standard error is the standard deviation of the sampling distribution, or the set of estimates that could have arisen under all possible random assignments. 
\item How is randomization inference used to test the sharp null hypothesis of no effect for any subject? 
Answer:\\ 
The sharp null hypothesis of no effect is a case in which $Y_i (1)= Y_i (0)$; under this assumption, all potential outcomes are observed because treated and untreated potential outcomes are identical.  In order to form the sampling distribution under the sharp null hypothesis of no effect, we simulate a random assignment and calculate the test statistic (for example, the difference-in-means between the assigned treatment and control groups). This simulation is repeated a large number of times in order to form the sampling distribution under the null hypothesis.  The $p$-value of the test statistic that is observed in the actual experiment is calculated by finding its location in the sampling distribution under the null hypothesis. For example, if the observed test statistic is as large or larger than 9,000 of 10,000 simulated experiments, the one-tailed $p$-value is 0.10.
\item What is a 95\% confidence interval?  \\
Answer:\\ 
A confidence interval consists of two estimates, a lower number and an upper number, that are intended to bracket the true parameter of interest with a specified probability. An estimated confidence interval is a random variable that varies from one experiment to the next due to random variability in how units are allocated to treatment and control. A 95\% interval is designed to bracket the true parameter with a 0.95 probability across hypothetical replications of a given experiment.  In other words, across hypothetical replications, 95\% of the estimated 95\% confidence intervals will bracket the true parameter.  
\item How does complete random assignment differ from block random assignment and clustered random assignment?
Answer:\\
Under complete random assignment, each subject is assigned separately to treatment or control groups such that m of N subjects end up in the treatment condition. Under block random assignment, complete random assignment occurs within each block or subgroup. Under clustered assignment, groups of subjects are assigned jointly to treatment or control; the assignment procedure requires that if one member of the group is assigned to the treatment group, all others in the same group are also assigned to treatment. 
\item Experiments that assign the same number of subjects to the treatment group and control group are said to have a ``balanced design.''  What are some desirable statistical properties of balanced designs?\\
Answer:\\
One desirable property of a balanced design is that under certain conditions, it generates less sampling variability than unbalanced designs; this property of balanced designs holds when the variance of $Y_i(0)$ is approximately the same as the variance of $Y_i (1)$. Another attractive property is that estimated confidence intervals are, on average, conservative (they tend to overestimate the true amount of sampling variability) under balanced designs. (A final attractive property, which comes up in Chapter 4, is that regression is less prone to bias under balanced designs.)
\end{enumerate}

\section*{Question 2}
Rewrite equation (3.4) substituting for $Y_i (1)$ using the equation $Y_i (1) = Y_i (0) + \tau_i$.  Assume that $N=2m$, and interpret the implications of the resulting formula for experimental design. [5 points]

Answer:\\
Substituting $N =2m$ and $Y_i (1) = Y_i (0) + \tau_i$ gives:
\begin{align*}
SE(\widehat{ATE}) &= \sqrt{\frac{1}{(N-1)} \left \{ \frac{mVar(Y_i (0))}{2m- m} + \frac{mVar(Y_i (0) + \tau_{i})}{2m- m}\right \} + 2cov(Y_i(0), Y_i(0) + \tau_{i})} \\
&= \sqrt{\frac{1}{(N-1)} \left \{ Var(Y_i (0)) + Var(Y_i (0) + \tau_{i}) + 2Var(Y_i(0)) + 2cov(Y_i(0), \tau_{i})\right \}} \\
&= \sqrt{\frac{1}{(N-1)} \left \{ 3Var(Y_i (0)) +[Var(Y_i (0)) + Var(\tau_{i}) + 2cov(Y_i(0), \tau_i)] + 2cov(Y_i(0), \tau_{i})\right \}}\\
&= \sqrt{\frac{1}{(2m-1)} \left \{ 4Var(Y_i (0)) + Var(\tau_{i}) + 4cov(Y_i(0), \tau_{i})\right \}}
\end{align*}
All else being equal, the true standard error is smaller when the variance of the treatment effect is smaller, the variance of $Y_i(0)$ is smaller, and the covariance of the treatment effect and $Y_i(0)$ is smaller.

\section*{Question 3}
Using the equation $Y_i (1) = Y_i (0) + \tau_i$, show that when we assume that treatment effects are the same for all subjects, 
$Var(Y_i(0))= Var(Y_i(1))$ and the correlation between $Y_i (0)$ and $Y_i (1)$ is 1.0.[5 points]

Under constant treatment effects,  $Var(Y_i (1)=Var(Y_i (0)+\tau)=Var(Y_i (0))$, and the correlation between $Y_i(1)$ and $Y_i(0)$ is:

\begin{align*}
cor(Y_i(1), Y_i(0)) &= \frac{Cov(Y_i(1), Y_i(0))}{\sqrt{Var(Y_i(1)) * Var(Y_i(0))}} \\
&=\frac{Cov(Y_i(0) + \tau, Y_i(0))}{\sqrt{Var(Y_i(0)) * Var(Y_i(0))}} \\
& = \frac{Var(Y_{i}(0))}{Var(Y_{i}(0))}\\
&= 1
\end{align*}

\section*{Question 4}
Consider the schedule of outcomes in the table below. If treatment A is administered, the potential outcome is $Y_i (A)$, and if treatment B is administered, the potential outcome is $Y_i (B)$.  If no treatment is administered, the potential outcome is $Y_i (0)$.  The treatment effects are defined as $Y_i (A)-Y_i (0)$ or $Y_i (B)-Y_i (0)$. [5 points]


% Table generated by Excel2LaTeX from sheet 'Sheet1'
\begin{table}[htbp]
  \centering
  \caption{Question 4 Table}
    \begin{tabular}{rccc}
    \toprule
    Subject &       &       &  \\
    \midrule
    Miriam & 1     & 2     & 3 \\
    Benjamin & 2     & 3     & 3 \\
    Helen & 3     & 4     & 3 \\
    Eva   & 4     & 5     & 3 \\
    Billie & 5     & 6     & 3 \\
    \bottomrule
    \end{tabular}%
  \label{tab:addlabel}%
\end{table}

Suppose a researcher plans to assign two observations to the control group and the remaining three observations to just one of the two treatment conditions.  The researcher is unsure which treatment to use.  

\begin{enumerate}[a)]
\item Applying equation (3.4), determine which treatment, A or B, will generate a sampling distribution with a smaller standard error.\\
Answer:\\
First, notice that $Y_i (A)=Y_i (0)+1$. Then using the results developed in the previous exercise:

\begin{align*}
SE(\widehat{ATE_{A}}) &= \sqrt{\frac{1}{5-1}\left \{ \frac{3*2}{2} + \frac{2*2}{3} + 2*2 \right \}} \\
&= 1.44\\
SE(\widehat{ATE_{B}}) &= \sqrt{\frac{1}{5-1}\left \{ \frac{3*2}{2} + \frac{2*0}{3} + 2*0 \right \}} \\
&= 0.86
\end{align*}

The standard error for the B vs. control comparison is smaller than the standard error for the A vs. control comparison. Thus, administering treatment B gives rise to a narrower sampling distribution.

\item What does the result in part (a) imply about the feasibility of studying interventions that attempt to close an existing ``achievement gap''?\\
Answer:\\
When treatment B is administered, the achievement gap between the best and worst student narrows, leaving no variance in $Y_i (B)$. Two of the three terms in equation (3.4) therefore become zero, and the resulting standard error is much lower than it would be under treatment A, which has a constant effect across all subjects. The basic principle here is that it helps to study treatments that reduce the covariance between untreated and treated potential outcomes. 
\end{enumerate}

\section*{Question 5}
Using Table 2.1, imagine that your experiment allocates one village to treatment. [10 points]

\begin{enumerate}[a)]
\item Calculate the estimated difference-in-means for all seven possible randomizations.\\
Answer:\\
There are 7 subjects, 1 of which is assigned to treatment, and thus the number of randomizations is $\frac{7!}{1!(7-1)!}=7$. Now let's define $\widehat{ATE_{i}}$ as the difference in means constructed when assuming village i is assigned to treatment.

% Table generated by Excel2LaTeX from sheet 'Sheet1'
\begin{table}[htbp]
  \centering
  \caption{Question 5 Table}
    \begin{tabular}{cccc|l}
    \toprule
     Village & $Y_i(0)$       & $Y_i(1)$    & $\tau_i$        & \multicolumn{1}{c}{$\widehat{ATE_{i}}$} \\
    \midrule
    1     &  10     &  15     & 5      &  $15 - \frac{15+20+20+10+15+15}{6} = -\frac{5}{6}$\\
    2     &  15    &  15     &  0     &  $15 - \frac{10+20+20+10+15+15}{6} = 0$ \\
    3     &  20     & 30      &  10     &  $30 - \frac{10+15+20+10+15+15}{6} = \frac{95}{6}$\\
    4     &   20    &   15    &   -5    &  $15 - \frac{10+15+20+10+15+15}{6} = \frac{5}{6}$\\
    5     &  10      &  20     &  10     &  $20 - \frac{10+15+20+20+15+15}{6} = \frac{25}{6}$\\
    6     & 15      &   15    &   0    & $15 - \frac{10+15+20+20+10+15}{6} = 0$ \\
    7     &   15    &   39    &   15    &  $30 - \frac{10+15+20+20+10+15}{6} = 15$ \\ \midrule
    Mean  &   15    &   20    &   5    & $\frac{-\frac{5}{6} + 0 + \frac{95}{6} + \frac{5}{6} + \frac{25}{6} + 0 + 15}{7} = 5$ \\
    SD    & $\sqrt{\frac{2(10-15)^{2} + 2(20-15)^{2}}{7}}$      &  $\sqrt{\frac{4(15-20)^{2} + 2(30-20)^{2}}{7}}$      &       & $\sqrt{\frac{(-\frac{5}{6}-5)^{2} + 2(-5)^{2} + (\frac{95}{6}-5)^{2} + (\frac{5}{6}-5)^{2}) + (\frac{25}{6}-5)^{2} + (15-5)^{2}}{7}}$ \\
    & $= \sqrt{\frac{100}{7}}$ & $= \sqrt{\frac{300}{7}}$ & &  $= 6.755$\\
    \bottomrule
    \end{tabular}%
  \label{tab:addlabel}%
\end{table}%

\item Show that the average of these estimates is the true ATE.\\
Answer:\\
The table shows that the average across all randomizations is 5, which is the true ATE.
\item Show that the standard deviation of the seven estimates is identical to the standard error implied by equation (3.4).  

Beginning with Equation 3.4:
\begin{align*}
SE(\widehat{ATE}) &= \sqrt{\frac{1}{(N-1)} \left \{ \frac{mVar(Y_i (0))}{N- m} + \frac{(N-m)*Var(Y_i (1))}{m} + 2cov(Y_i(0), Y_i(1))\right \}} \\
&= \sqrt{\frac{1}{6} \left \{ \frac{Var(Y_i (0))}{6} + 6Var(Y_i (1)) + 2cov(Y_i(0), Y_i(1))\right \}} \\
cov(Y_i(0), Y_i(1)) &= \frac{(10-15)(15-20) + (20-15)(30-20) + (20-15)(15-20)}{7} = \frac{50}{7}\\
&= \sqrt{\frac{1}{6} \left \{ \frac{\frac{100}{7}}{6} + 6\frac{300}{7} + 2\frac{50}{7}\right \}} \\
&= 6.755
\end{align*}

This is identical to the standard deviation calculated in the table above.

\item Referring to equation (3.4), explain why this experimental design has more sampling variability than the design in which two villages out of seven are assigned to treatment.\\
Answer:\\
The covariance term is unaffected, but the first two variance terms are multiplied by different numbers. The first term is multiplied by 1/6 in this example as opposed to 2/5 in the 2-of-7 example. The second term is multiplied by 6/1 in this example as opposed to 5/2 in the 2-of-7 example. Because the second variance term is larger than the first, allocating more sample to the treatment group reduces sampling variance.

\begin{align*}
SE(\widehat{ATE}) &= \sqrt{\frac{1}{(N-1)} \left \{ \frac{mVar(Y_i (0))}{N- m} + \frac{(N-m)*Var(Y_i (1))}{m} + 2cov(Y_i(0), Y_i(1))\right \}} \\
&= \sqrt{\frac{1}{6} \left \{ \frac{1}{6}\frac{100}{7} + \frac{6}{1}\frac{300}{7} + 2\frac{50}{7}\right \}} = 6.755 \text{, if $m = 1$} \\
&= \sqrt{\frac{1}{6} \left \{ \frac{2}{5}\frac{100}{7} + \frac{5}{2}\frac{300}{7} + 2\frac{50}{7}\right \}} = 4.603 \text{, if $m = 2$} \\
\end{align*}

\item Explain why, in this example, a design in which one of seven observations is assigned to treatment has more\footnote{Text mistakenly printed ``less''} sampling variability than a design in which six villages out of seven are assigned to treatment.  

\begin{align*}
SE(\widehat{ATE}) &= \sqrt{\frac{1}{(N-1)} \left \{ \frac{mVar(Y_i (0))}{N- m} + \frac{(N-m)*Var(Y_i (1))}{m} + 2cov(Y_i(0), Y_i(1))\right \}} \\
&= \sqrt{\frac{1}{6} \left \{ \frac{1}{6}\frac{100}{7} + \frac{6}{1}\frac{300}{7} + 2\frac{50}{7}\right \}} = 6.755 \text{, if $m = 1$} \\
&= \sqrt{\frac{1}{6} \left \{ \frac{6}{1}\frac{100}{7} + \frac{1}{6}\frac{300}{7} + 2\frac{50}{7}\right \}} = 4.23 \text{, if $m = 6$} \\
\end{align*}

By the same logic as above -- allocating more units to the condition in which potential outcomes are more variable can reduce sampling variability.

\end{enumerate}

\section*{Question 6}
The Clingingsmith, Khwaja, and Kremer study discussed in section 3.5 may be used to test the sharp null hypothesis that winning the visa lottery for the pilgrimage to Mecca had no effect on the views of Pakistani Muslims toward people from other countries.Assume that the visa authorities conducted a complete random assignment; generate 10,000 simulated random assignments under the sharp null hypothesis.  How many of the simulated random assignments generate an estimated ATE that is at least as large as the actual estimate of the ATE? What is the implied one-tailed p-value? How many of the simulated random assignments generate an estimated ATE that is at least as large in absolute value as the actual estimate of the ATE? What is the implied two-tailed p-value? [10 points]

\begin{lstlisting}[language=stata]
. // download data from: http://hdl.handle.net/10079/6hdr852
. // copy and paste the url to your web browser
. 
. import delim "Clingingsmith_et_al_QJE_2009dta.csv",clear
(8 vars, 958 obs)

. set seed 1234567
. rename success D
. rename views Y
//findit tsrtest
. //package name:  st0158.pkg install
. 
. cap program drop ate
. program define ate, rclass
  1.         args Y D
  2.     sum `Y' if `D'==1, meanonly
  3.     local Y_treat=r(mean)
  4.     sum `Y' if `D'==0, meanonly
  5.     local Y_con=r(mean)
  6.     return scalar ate_avg = `Y_treat'-`Y_con'
  7. end

. // ssc install tsrtest
. tsrtest D r(ate_avg) using 3_6_resam.dta, overwrite: ate Y D
Two-sample randomization test for theta=r(ate_avg) of ate Y D by D

Combinations:   8.4503047638e+285 = (958 choose 448)
Assuming null=0
Observed theta: .4748

Minimum time needed for exact test (h:m:s):  4.2e+278:00:00
Reverting to Monte Carlo simulation.
Mode: simulation (10000 repetitions)

progress: |........................................|

 p=0.00190 [one-tailed test of Ho:  theta(D==0)<=theta(D==1)]
 p=0.99830 [one-tailed test of Ho:  theta(D==0)>=theta(D==1)]
 p=0.00360 [two-tailed test of Ho:  theta(D==0)==theta(D==1)]

Saving log file to 3_6_resam.dta...done.
.   
. preserve 
. use "3_6_resam.dta", clear

. global ate = theta[1]

. di $ate
.4748337

. drop if _n==1
(1 observation deleted)

. count if theta >= $ate
  19

. scalar p_onesided = r(N)/_N

. count if abs(theta) >= $ate
  36

. scalar p_twosided = r(N)/_N

. di "p.value.onesided = "p_onesided
p.value.onesided = .0019

. di "p.value.twosided = "p_twosided 
p.value.twosided = .0036

. restore
\end{lstlisting}


The estimated ATE is 0.4748337.  The number of simulated ATEs under the sharp null hypothesis of no effect that were as large was 15, corresponding to a $p$-value of 0.0017.  The number of simulated ATEs under the sharp null hypothesis of no effect that were as large in absolute value was  was 32, corresponding to a $p$-value of 0.0034.

\section*{Question 7}
A diet and exercise program advertises that it causes everyone who is currently dieting to lose at least seven pounds more than they otherwise would have during the first two weeks.  Use randomization inference (the procedure described in section 3.4) to test the hypothesis that $\tau_i=7$ for all $i$.  The treatment group's weight losses after two weeks are (2, 11, 14, 0, 3) and the control group's weight losses are (1, 0, 0, 4, 3).  In order to test the hypothesis $\tau_i=7$ for all $i$ using the randomization inference methods discussed in this chapter, subtract 7 from each outcome in the treatment group so that the exercise turns into the more familiar test of the sharp null hypothesis that $\tau_i=0$ for all $i$. When describing your results, remember to state the null hypothesis clearly, and explain why you chose to use a one-sided or two-sided test. [10 points]

% Table generated by Excel2LaTeX from sheet 'Sheet1'
\begin{table}[htbp]
  \centering
  \caption{Question 7 Table}
    \begin{tabular}{rrrr}
    \toprule
     Subject & $Y_i(0)$      & $Y_i(1)$       & $Y_i(1) - 7$  \\
    \midrule
    1     &   ?    &   2    &  -5 \\
    2     &   ?    &  11     &  4 \\
    3     &   ?    &  14     &  7 \\
    4     &   ?    &  0     &   -7\\
    5     &   ?    &  3     &   -4\\
    6     &   1    &  ?     &   ?\\
    7     &   0    &  ?     & ?  \\
    8     &   0    &  ?     & ?  \\
    9     &   4    &  ?     & ?  \\
    10    &   3    &  ?     & ? \\
    \bottomrule
    \end{tabular}%
  \label{tab:addlabel}%
\end{table}

\begin{lstlisting}[language=stata]
. clear
. set seed 1234567
. set obs 10
number of observations (_N) was 0, now 10
. 
. input D Y

             D          Y
  1. 0 1 
  2. 0 0 
  3. 0 0 
  4. 0 4 
  5. 0 3 
  6. 1 2 
  7. 1 11 
  8. 1 14 
  9. 1 0 
 10. 1 3 
. 
. gen Y_star= Y+D*(-7)
. 
. cap program drop ate

. program define ate, rclass
  1.         args Y D
  2.         sum `Y' if `D'==1, meanonly
  3.         local Y_treat=r(mean)
  4.         sum `Y' if `D'==0, meanonly
  5.         local Y_con=r(mean)
  6.         return scalar ate_avg = `Y_treat'-`Y_con'
  7. end
. 
. // findit tsrtest (to install the package)
. tsrtest D r(ate_avg): ate Y_star D
Two-sample randomization test for theta=r(ate_avg) of ate Y_star D by D

Combinations:   252 = (10 choose 5)
Assuming null=0
Observed theta: -2.6

Minimum time needed for exact test (h:m:s):  0:00:00
Mode: exact

progress: |........................................|

 p=0.83730 [one-tailed test of Ho:  theta(D==0)<=theta(D==1)]
 p=0.20635 [one-tailed test of Ho:  theta(D==0)>=theta(D==1)]
 p=0.41270 [two-tailed test of Ho:  theta(D==0)==theta(D==1)]
. 
. // ate
. di r(obsvStat)       
-2.6

. 
. // p.value.onesided
. di r(lowertail)   
.20634921

\end{lstlisting}

There are 10 subjects, 5 of which are assigned to treatment, and thus the number of randomizations is $\frac{10!}{5!5!}=252$.  The null hypothesis is that the true ATE is a 7 pound loss; the alternative hypothesis is that the weight loss ATE is less than 7 pounds.  A one-sided hypothesis test is used because we only want to reject the weight loss program's claims if the observed weight loss is less than what they claimed; if they understated the degree of weight loss, their program would be even more effective than claimed, and one would hardly fault them for that.  Using the code for randomization inference posted on the website, we find that the observed difference in weight loss between the treatment and control groups (6 - 1.6 = 4.4) is smaller than 79\% of all simulated experiments under the null hypothesis of a 7 pound effect for everyone.  Thus, the p-value is 0.21, meaning we cannot reject the null hypothesis of a 7-pound effect at the conventional 0.05 significance threshold.

\section{Question 8}


Natural experiments sometimes involve what is, in effect, block random assignment. For example, Titiunik studies the effect of lotteries that determine whether state senators in Texas and Arkansas serve two-year or four-year terms in the aftermath of decennial redistricting.\footnote{Titiunik 2010.} These lotteries are conducted within each state, and so there are effectively two distinct experiments on the effects of term length. An interesting outcome variable is the number of bills (legislative proposals) that each senator introduces during a legislative session. The table below lists the number of bills introduced by senators in each state during 2003. [10 points]

% Table generated by Excel2LaTeX from sheet 'Sheet1'
\begin{table}[htbp]
  \centering
  \caption{Question 8 Table}
    \begin{tabular}{C{4cm}C{2cm}|C{4cm}C{2cm}}
    \toprule
    \multicolumn{2}{c}{Texas} & \multicolumn{2}{c}{Arkansas} \\
    \midrule
    Term Length: 0 = four-year term; 1 = two-year term & \# of bills introduced & Term Length: 0 = four-year term; 1 = two-year term & \# of bills introduced \\
    0     & 18    & 0     & 11 \\
    0     & 29    & 0     & 15 \\
    0     & 41    & 0     & 17 \\
    0     & 53    & 0     & 23 \\
    0     & 60    & 0     & 24 \\
    0     & 67    & 0     & 25 \\
    0     & 75    & 0     & 26 \\
    0     & 79    & 0     & 28 \\
    0     & 79    & 0     & 31 \\
    0     & 88    & 0     & 33 \\
    0     & 93    & 0     & 34 \\
    0     & 101   & 0     & 35 \\
    0     & 103   & 0     & 35 \\
    0     & 106   & 0     & 36 \\
    0     & 107   & 0     & 38 \\
    0     & 131   & 0     & 52 \\
    1     & 29    & 0     & 59 \\
    1     & 37    & 1     & 9 \\
    1     & 42    & 1     & 10 \\
    1     & 45    & 1     & 14 \\
    1     & 45    & 1     & 15 \\
    1     & 54    & 1     & 15 \\
    1     & 54    & 1     & 17 \\
    1     & 58    & 1     & 18 \\
    1     & 61    & 1     & 19 \\
    1     & 64    & 1     & 19 \\
    1     & 69    & 1     & 20 \\
    1     & 73    & 1     & 21 \\
    1     & 75    & 1     & 23 \\
    1     & 92    & 1     & 23 \\
    1     & 104   & 1     & 24 \\
          &       & 1     & 28 \\
          &       & 1     & 30 \\
          &       & 1     & 32 \\
          &       & 1     & 34 \\
    \bottomrule
    \end{tabular}%
  \label{tab:addlabel}%
\end{table}%

\begin{enumerate}[a)]
\item For each state, estimate of the effect of having a two-year term on the number of bills introduced.

\begin{lstlisting}[language=stata]
 // download data from : http://hdl.handle.net/10079/s1rn910
. // copy and paste the url to your web browser
. use "Titiunik_WorkingPaper_2010.csv.dta",clear 
. 
. set seed 1234567

.         rename term2year D
.         rename bills_introduced Y
.         rename texas0_arkansas1 block         
.         qui tabstat Y if block ==0, by(D) stat(mean) save       
.         scalar ate_texas = el(r(Stat2),1,1) - el(r(Stat1),1,1)         
.         qui tabstat Y if block ==1, by(D) stat(mean) save       
.         scalar ate_ark = el(r(Stat2),1,1) - el(r(Stat1),1,1)
.         
.         di "ate_texas="%18.5f ate_texas 
ate_texas=         -16.74167

.         di "ate_arkansas="%18.5f ate_ark        
ate_arkansas=         -10.09477

\end{lstlisting}


The estimated ATE in Texas is \ensuremath{-16.742}.  In Arkansas, the estimated ATE is \ensuremath{-10.095}.

\item For each state, estimate the standard error of the estimated ATE.

\begin{lstlisting}[language=stata]

 qui tabstat Y if block ==0, by(D) stat(v n) save        
. scalar se_texas = sqrt(el(r(Stat2),1,1)/el(r(Stat2),2,1) + /// 
>                                         el(r(Stat1),1,1)/el(r(Stat1),2,1))
.                                         
. 
. qui tabstat Y if block ==1, by(D) stat(v n) save        

. 
. scalar se_arkansas = sqrt(el(r(Stat2),1,1)/el(r(Stat2),2,1) + /// 
>                                         el(r(Stat1),1,1)/el(r(Stat1),2,1)) 
. 
. di "se_texas="%18.6f se_texas
se_texas=          9.345871

. di "se_arkansas="%18.6f se_arkansas
se_arkansas=          3.395979

\end{lstlisting}

The estimated se in Texas is 9.346.  In Arkansas, the estimated se is 3.396.

\item Use equation (3.10) to estimate the overall ATE for both states combined.

\begin{lstlisting}[language=stata]
 qui tabstat Y, by(block) stat(n) save   
. 
. scalar ate_overall = el(r(Stat1),1,1)/_N*ate_texas + /// 
>                                          el(r(Stat2),1,1)/_N*ate_ark
. 
. 
. di %18.4f ate_overall
          -13.2168
. 
. // same as
. // teffects nnmatch (bills_introduced) (term2year), ematch(texas0_arkansas1)

\end{lstlisting}


The overall ATE, \ensuremath{-13.217} is the weighted average of the two separate ATEs, where the weights are the shares of overall $N$ in each state.

\item Explain why, in this study, simply pooling the data for the two states and comparing the average number of bills introduced by two-year senators to the average number of bills introduced by four-year senators leads to biased estimates of the overall ATE. \\
Answer:\\
The two states differ in terms of the probability that a given legislator will be assigned to the treatment. Therefore, we cannot pool the data without introducing a correlation between treatment assignment and the potential outcomes associated with the two states. In this study, the experiments take place within each state, and the analyst should pool the state-level results in order to obtain an overall result.

\item Insert the estimated standard errors into equation (3.12) to estimate the standard error for the overall ATE.

\begin{lstlisting}[language=stata]
. scalar se_overall = sqrt((el(r(Stat1),1,1)/_N)^2*se_texas^2 + /// 
>                                          (el(r(Stat2),1,1)/_N)^2*se_arkansas^2)
.                                          
. di %18.5f se_overall
           4.74478

\end{lstlisting}

The overall standard error is (4.745).

\item Use randomization inference to test the sharp null hypothesis that the treatment effect is zero for senators in both states.

\begin{lstlisting}[language=stata]
 // calculate probs under block assignment
. bysort block: egen probs=mean(D). 
. 
. 
. cap program drop ate_block

. 
. program define ate_block, rclass
  1. args Y D probs
  2. tempvar ipw
  3. gen `ipw' = .
  4. // calculate inverse probability weight under block assignment
. replace `ipw' = `D'/`probs' + (1-`D')/(1-`probs')
  5. qui reg `Y' `D' [iw=`ipw']
  6. return scalar ate=_b[`D']
  7. end 
. 
. // ssc install ritest (to install ritest package)
. 
. //
. ritest D r(ate), strata(block) reps(10000) nodots: ///
> ate_block Y D probs
(66 missing values generated)
(66 real changes made)

      command:  ate_block Y D probs
        _pm_1:  r(ate)
  res. var(s):  D
   Resampling:  Permuting D
Clust. var(s):  __000000
     Clusters:  66
Strata var(s):  block
       Strata:  2

------------------------------------------------------------------------------
T            |     T(obs)       c       n   p=c/n   SE(p) [95% Conf. Interval]
-------------+----------------------------------------------------------------
       _pm_1 |   -13.2168      65   10000  0.0065  0.0008    .00502   .0082774
------------------------------------------------------------------------------
Note: Confidence interval is with respect to p=c/n.
Note: c = #{|T| >= |T(obs)|}

. // ate
. di el(r(b),1,1)
-13.216796

. 
. // p.value.twosided
. di el(r(p),1,1)
.0065
\end{lstlisting}

Here, we use a two-tailed test because it is not clear theoretically whether longer or shorter terms should make legislators more responsive. Comparing the observed difference-in-means to the distribution of 10,000 simulated randomizations under the sharp null hypothesis reveals a two-tailed p-value of 0.0071, leading us to reject the null hypothesis. 
\end{enumerate}

\section*{Question 9}

Camerer reports the results of an experiment in which he tests whether large, early bets placed at horse tracks affect the betting behavior of other bettors.\footnote{Camerer 1998.  This example draws on the second of Camerer's studies and restricts the sample to cases in which a treatment horse is compared to a single control horse.} Selecting pairs of long-shot horses running in the same race whose betting odds were approximately the same when betting opened, he placed two \$500 bets on one of the two horses approximately 15 minutes before the start of the race. Because odds are determined based on the proportion of total bets placed on each horse, this intervention causes the betting odds for the treatment horse to decline and the betting odds of the control horse to rise. Because Camerer's bets were placed early, when the total betting pool was small, his bets caused marked changes in the odds presented to other bettors. (A few minutes before each race started, Camerer canceled his bets.) While the experimental bets were still ``live,'' were other bettors attracted to the treatment horse (because other bettors seemed to believe in the horse) or repelled by it (because the diminished odds meant a lower return for each wager)? Seventeen pairs of horses in this study are listed below. The outcome measure is the number of dollars that were placed on each horse (not counting Camerer's own wagers on the treatment horses) during the test period, which begins 16 minutes before each race (roughly 2 minutes before Camerer began placing his bets) and ends 5 minutes before each race (roughly 2 minutes before Camerer withdrew his bets). [10 points]

% Table generated by Excel2LaTeX from sheet 'Sheet1'
\begin{table}[H]
  \centering
  \caption{Question 9 Table}
    \begin{tabular}{rR{2cm}R{1.8cm}R{1.5cm}R{2cm}R{1.8cm}R{1.5cm}R{1.8cm}}
    \toprule
    & \multicolumn{3}{c}{Treatment Horse in Pair}& \multicolumn{3}{c}{Control Horse in Pair} \\ \cmidrule(r){2-4}\cmidrule(r){5-7}
          & Total bets $T-16$ min & Total bets $T-5$ min & Change & Total bets $T-16$ min & Total bets $T-5$ min & Change & Difference in changes \\
    \midrule
    Pair 1 & 533   & 1503  & 970   & 587   & 2617  & 2030  & -1060 \\
    Pair 2 & 376   & 1186  & 810   & 345   & 1106  & 761   & 49 \\
    Pair 3 & 576   & 1366  & 790   & 653   & 2413  & 1760  & -970 \\
    Pair 4 & 1135  & 1666  & 531   & 1296  & 2260  & 964   & -433 \\
    Pair 5 & 158   & 367   & 209   & 201   & 574   & 373   & -164 \\
    Pair 6 & 282   & 542   & 260   & 269   & 489   & 220   & 40 \\
    Pair 7 & 909   & 1597  & 688   & 775   & 1825  & 1050  & -362 \\
    Pair 8 & 566   & 933   & 367   & 629   & 1178  & 549   & -182 \\
    Pair 9 & 0     & 555   & 555   & 0     & 355   & 355   & 200 \\
    Pair 10 & 330   & 786   & 456   & 233   & 842   & 609   & -153 \\
    Pair 11 & 74    & 959   & 885   & 130   & 256   & 126   & 759 \\
    Pair 12 & 138   & 319   & 181   & 179   & 356   & 177   & 4 \\
    Pair 13 & 347   & 812   & 465   & 382   & 604   & 222   & 243 \\
    Pair 14 & 169   & 329   & 160   & 165   & 355   & 190   & -30 \\
    Pair 15 & 41    & 297   & 256   & 33    & 75    & 42    & 214 \\
    Pair 16 & 37    & 71    & 34    & 33    & 121   & 88    & -54 \\
    Pair 17 & 261   & 485   & 224   & 282   & 480   & 198   & 26 \\
    \bottomrule
    \end{tabular}%
  \label{tab:addlabel}%
\end{table}

\begin{enumerate}[a)]
\item One interesting feature of this study is that each pair of horses ran in the same race.  Does this design feature violate the non-interference assumption, or can potential outcomes be defined so that the non-interference assumption is satisfied? \\
Answer:\\
This design feature violates non-interference if the estimand is defined as the difference between the following two potential outcomes: total bets on a given horse when experimental bets are placed on that horse versus no experimental bets on any horse in the race.  One could avoid violating non-interference by redefining the estimand as the difference between the following two potential outcomes: total bets on a horse when experimental bets are placed on that horse versus experimental bets are placed on a competing horse in the same race. 

\item A researcher interested in conducting a randomization check might assess whether, as expected, treatment and control horses attract similarly sized bets prior to the experimental intervention.  Use randomization inference to test the sharp null hypothesis that the bets had no effect prior to being placed. \\

\begin{lstlisting}[language=stata]
 // download data from : http://hdl.handle.net/10079/1g1jx43
. // copy and paste the url to your web browser
. 
. use "Camerer_JPEsubset_1998.dta.dta", clear 
. 
. set seed 1234567

.         rename treatment D

.         rename pair block

.         rename preexperimentbets covs
. 
.         // calculate probs under block assignment
.         bysort block: egen probs=mean(D)

.         
.                 
.         // permuation to calculate F stat and one-side P value
.         ritest D e(F), strata(block) reps(10000) right nodots: ///
>         regress D covs

      Source |       SS           df       MS      Number of obs   =        34
-------------+----------------------------------   F(1, 32)        =      0.02
       Model |  .005024372         1  .005024372   Prob > F        =    0.8914
    Residual |  8.49497563        32  .265467988   R-squared       =    0.0006
-------------+----------------------------------   Adj R-squared   =   -0.0306
       Total |         8.5        33  .257575758   Root MSE        =    .51524

------------------------------------------------------------------------------
           D |      Coef.   Std. Err.      t    P>|t|     [95% Conf. Interval]
-------------+----------------------------------------------------------------
        covs |  -.0000386   .0002809    -0.14   0.891    -.0006109    .0005336
       _cons |   .5137818   .1335793     3.85   0.001     .2416896     .785874
------------------------------------------------------------------------------

      command:  regress D covs
        _pm_1:  e(F)
  res. var(s):  D
   Resampling:  Permuting D
Clust. var(s):  __000000
     Clusters:  34
Strata var(s):  block
       Strata:  17

------------------------------------------------------------------------------
T            |     T(obs)       c       n   p=c/n   SE(p) [95% Conf. Interval]
-------------+----------------------------------------------------------------
       _pm_1 |   .0189265    3736   10000  0.3736  0.0048  .3641064   .3831672
------------------------------------------------------------------------------
Note: Confidence interval is with respect to p=c/n.
Note: c = #{T >= T(obs)}

. 
.         // p.value
.         di el(r(p),1,1)
.3736



\end{lstlisting}

We conducted 10,000 random assignments, and for each we calculated the F-statistic of a regression of treatment assignment on pre-experimental bets (controlling for blocks).  The observed F-statistic for the actual experiment is larger than 3696 of the simulated experiments, implying a p-value of 0.37.

\item Calculate the average increase in bets during the experimental period for treatment horses and control horses.  Compare treatment and control means, and interpret the estimated ATE.  

\begin{lstlisting}[language=stata]
. rename experimentbets change

. 
. tabstat change, by(D) stat(mean) save   

Summary for variables: change
     by categories of: D 

       D |      mean
---------+----------
       0 |  571.4118
       1 |  461.2353
---------+----------
   Total |  516.3235
--------------------

. 
. di "ATE ="%180.4f el(r(Stat2),1,1)-el(r(Stat1),1,1)
ATE =                 -110.1765


\end{lstlisting}


The average treatment group change was \$461.24, as opposed to an average change of \$571.41 in the control group.  Therefore, the estimated ATE is \$\ensuremath{-110.18}.
\item Show that the estimated ATE is the same when you subtract the control group outcome from the treatment group outcome for each pair and calculate the average difference for the 17 pairs. 
Answer:\\

\begin{lstlisting}[language=stata]
 bysort block (D): gen pair_diff = change - change[_n+1]
(17 missing values generated)

. mean(pair_diff)

Mean estimation                   Number of obs   =         17

--------------------------------------------------------------
             |       Mean   Std. Err.     [95% Conf. Interval]
-------------+------------------------------------------------
   pair_diff |   110.1765   104.8377     -112.0695    332.4225
--------------------------------------------------------------

. 
. // the same as
. // teffects nnmatch (experimentbets block) (D)

\end{lstlisting}

The average difference between treatment and control outcomes for each pair is also 110.18.

\item Use randomization inference to test the sharp null hypothesis of no treatment effect for any subject.  When setting up the test, remember to construct the simulation to account for the fact that random assignment takes place within each pair.  Interpret the results of your hypothesis test and explain why a two-tailed test is appropriate in this application.

\begin{lstlisting}[language=stata]

. cap program drop ate_block

. 
. program define ate_block, rclass
  1. args Y D probs
  2. tempvar ipw
  3. gen `ipw' = .
  4. // calculate inverse probability weight under block assignment
. replace `ipw' = `D'/`probs' + (1-`D')/(1-`probs')
  5. qui reg `Y' `D' [iw=`ipw']
  6. return scalar ate=_b[`D']
  7. end 

. 
. 
. ritest D r(ate), strata(block) reps(10000) nodots: ///
> ate_block change D probs
(34 missing values generated)
(34 real changes made)

      command:  ate_block change D probs
        _pm_1:  r(ate)
  res. var(s):  D
   Resampling:  Permuting D
Clust. var(s):  __000000
     Clusters:  34
Strata var(s):  block
       Strata:  17

------------------------------------------------------------------------------
T            |     T(obs)       c       n   p=c/n   SE(p) [95% Conf. Interval]
-------------+----------------------------------------------------------------
       _pm_1 |  -110.1765    3170   10000  0.3170  0.0047  .3078845   .3262222
------------------------------------------------------------------------------
Note: Confidence interval is with respect to p=c/n.
Note: c = #{|T| >= |T(obs)|}. 
. 
. // ate
. di el(r(b),1,1)
-110.17647

. 
. // p.value.twosided
. di el(r(p),1,1)
.317
\end{lstlisting}


A two-tailed test generates a p-value of 0.3092, indicating that one cannot reject the sharp null of no effect for any unit. A two-tailed test is appropriate because some theories predict a positive effect while others predict a negative effect: ``were other bettors attracted to the treatment horse (because other bettors seemed to believe in the horse) or repelled by it (because the diminished odds meant a lower return for each wager)?''  The appropriate null hypothesis in this case is no effect, which would be rejected if we observed either strongly positive or strongly negative differences between treatment and control horses.
\end{enumerate}

\section*{Question 10}
Suppose that 800 individual students were randomly assigned to classrooms of 25 students apiece, and these classrooms were then randomly assigned as clusters to treatment and control. Assume the non-interference assumption holds. Use equations (3.4) and (3.22) to explain why this clustered design has the same standard error as complete random assignment of individual students to treatment and control. [10 points]
Answer:\\

The equation for the standard error under individual assignment:

\begin{equation*}
SE(\widehat{ATE}) = \sqrt{\frac{1}{(N-1)} \left \{ \frac{mVar(Y_i (0))}{N- m} + \frac{mVar(Y_i (1))}{N- m} + 2cov(Y_i(0), Y_i(1))\right \}}
\end{equation*}

The equation for the standard error under clustered assignment with equal-size clusters:

\begin{equation*}
SE(\widehat{ATE}) = \sqrt{\frac{1}{(k-1)} \left \{ \frac{mVar(\bar{Y}_j (0))}{N- m} + \frac{mVar(\bar{Y}_j (1))}{N- m} + 2cov(\bar{Y}_j(0), \bar{Y}_j(1))\right \}}
\end{equation*}
When the clusters are formed randomly (i.e., individuals are randomly allocated to clusters prior to assignment), the two formulas give approximately the same answer. In order to see the correspondence, notice that the variance of the average treated outcome from random draw of 25 students is $Var(\bar{Y}_j(0)) = \frac{Var(Y_i (0))}{25}$, and similarly, $Var(\bar{Y}_j(1)) = \frac{Var(Y_i (1))}{25}$, and $cov(\bar{Y}_j(0), \bar{Y}_j(1)) = \frac{cov(Y_i(0), Y_i(1))}{25}$. Thus, the quantity inside the braces in both equations differs by a factor of 25, which is approximately $\frac{N-1}{k-1}$.

\section*{Question 11}

Use the data in Table 3.3 to simulate cluster randomized assignment. [10 points]

\begin{enumerate}[a)]
\item Suppose that clusters are formed by grouping observations $\{1,2\},\{3,4\},\{5,6\}\ldots\{13,14\}$.  Use equation (3.22) to calculate the standard error assuming half of the clusters are randomly assigned to the treatment.

\begin{lstlisting}[language=stata]
. clear

. set seed 1234567

. set obs 14
number of observations (_N) was 0, now 14

. input Y0

            Y0
  1. 0
  2. 1
  3. 2
  4. 4
  5. 4
  6. 6
  7. 6
  8. 9
  9. 14
 10. 15
 11. 16
 12. 16
 13. 17
 14. 18

. end


. input Y1

            Y1
  1. 0
  2. 0
  3. 1
  4. 2
  5. 0
  6. 0
  7. 2
  8. 3
  9. 12
 10. 9
 11. 8
 12. 15
 13. 5
 14. 17

. end


. gen int cluster = (_n+1)/2

. 
. //ssc install tabstatmat  (install the package)
. // save tabstat summary result to matrix
. tabstat Y0, by(cluster) stat(mean) save

Summary for variables: Y0
     by categories of: cluster 

 cluster |      mean
---------+----------
       1 |        .5
       2 |         3
       3 |         5
       4 |       7.5
       5 |      14.5
       6 |        16
       7 |      17.5
---------+----------
   Total |  9.142857
--------------------

. tabstatmat Ybar0, nototal

Ybar0[7,1]
          Y0
1:mean    .5
2:mean     3
3:mean     5
4:mean   7.5
5:mean  14.5
6:mean    16
7:mean  17.5

. mat colnames Ybar0=Ybar0

. 
. tabstat Y1, by(cluster) stat(mean) save

Summary for variables: Y1
     by categories of: cluster 

 cluster |      mean
---------+----------
       1 |         0
       2 |       1.5
       3 |         0
       4 |       2.5
       5 |      10.5
       6 |      11.5
       7 |        11
---------+----------
   Total |  5.285714
--------------------

. tabstatmat Ybar1, nototal

Ybar1[7,1]
          Y1
1:mean     0
2:mean   1.5
3:mean     0
4:mean   2.5
5:mean  10.5
6:mean  11.5
7:mean    11

. mat colnames Ybar1=Ybar1 
. 
. 
. // function to calculate population variance
. cap program drop var_pop

. program define var_pop, rclass
  1.         args varname    
  2.         tempvar x_dev 
  3.         qui sum `varname'
  4.         local avg = r(mean)
  5.         local length = r(N)     
  6.         gen `x_dev' = (`varname'-`avg')^2/`length'
  7.         qui tabstat `x_dev', stat(sum) save
  8.         return scalar variance_pop = el(r(StatTotal),1,1)
  9. end
. 
. 
. // function to calculate population covariance
. cap program drop cor_pop

. program define cor_pop, rclass
  1.         args x y        
  2.         tempvar xy_dev 
  3.         qui sum `x'
  4.         local avg_x = r(mean)
  5.         local length = r(N)     
  6.         
.         qui sum `y'
  7.         local avg_y = r(mean)
  8.                 
.         gen `xy_dev' = (`x'-`avg_x')*(`y'-`avg_y')
  9.         qui tabstat `xy_dev', stat(sum) save
 10.         return scalar cor_pop = el(r(StatTotal),1,1)/`length'
 11. end

. 
. preserve 

. clear

. set obs 7
number of observations (_N) was 0, now 7

. svmat Ybar0, names(col)
number of observations will be reset to 7
Press any key to continue, or Break to abort
number of observations (_N) was 0, now 7

. svmat Ybar1, names(col)
. 
. // var_Ybar0    
. var_pop Ybar0

. scalar var_Ybar0=r(variance_pop)
. 
. // var_Ybar1 
. var_pop Ybar1

. scalar var_Ybar1=r(variance_pop)
. 
. // cov_Ybar0 
. cor_pop Ybar0 Ybar1

. 
. scalar cov_Ybar0=r(cor_pop)

. 
. scalar se_ate = sqrt((1/6)*((4/3)*var_Ybar0+(3/4)*var_Ybar1+2*cov_Ybar0))

. 
. di %8.6f se_ate
4.706192
. 
. restore
\end{lstlisting}







Assuming that 4 out of 7 clusters are assigned to treatment, the standard error of the ATE will be 4.71.

\item Suppose that clusters are instead formed by grouping observations $\{1,14\},\{2,13\},\{3,12\}\ldots\{7,8\}$. Use equation (3.22) to calculate the standard error assuming half of the clusters are randomly assigned to the treatment.

\begin{lstlisting}[language=stata]
 replace cluster = _n
(7 real changes made)

. replace cluster = 15-cluster if (cluster>7)
(7 real changes made)
. 
.         
. clear matrix

. // Ybar0        
. tabstat Y0, by(cluster) stat(mean) save

Summary for variables: Y0
     by categories of: cluster 

 cluster |      mean
---------+----------
       1 |         9
       2 |         9
       3 |         9
       4 |        10
       5 |       9.5
       6 |        10
       7 |       7.5
---------+----------
   Total |  9.142857
--------------------

. tabstatmat Ybar0, nototal

Ybar0[7,1]
         Y0
1:mean    9
2:mean    9
3:mean    9
4:mean   10
5:mean  9.5
6:mean   10
7:mean  7.5

. mat colnames Ybar0=Ybar0

. 
. // Ybar1
. tabstat Y1, by(cluster) stat(mean) save

Summary for variables: Y1
     by categories of: cluster 

 cluster |      mean
---------+----------
       1 |       8.5
       2 |       2.5
       3 |         8
       4 |         5
       5 |       4.5
       6 |         6
       7 |       2.5
---------+----------
   Total |  5.285714
--------------------

. tabstatmat Ybar1, nototal

Ybar1[7,1]
         Y1
1:mean  8.5
2:mean  2.5
3:mean    8
4:mean    5
5:mean  4.5
6:mean    6
7:mean  2.5

. mat colnames Ybar1=Ybar1
        
.         
. preserve 

. clear

. set obs 7
number of observations (_N) was 0, now 7

. svmat Ybar0, names(col)
number of observations will be reset to 7
Press any key to continue, or Break to abort
number of observations (_N) was 0, now 7

. svmat Ybar1, names(col) 

.         
. // var_Ybar0 <- var.pop(Ybar0)  
. var_pop Ybar0

. scalar var_Ybar0=r(variance_pop)

. 
. // var_Ybar1 <- var.pop(Ybar1)
. var_pop Ybar1

. scalar var_Ybar1=r(variance_pop)

. 
. // cov_Ybar0 <- cov.pop(Ybar0,Ybar1)
. cor_pop Ybar0 Ybar1

. scalar cov_Ybar0=r(cor_pop)

. 
. // se_ate
. scalar se_ate = sqrt((1/6)*((4/3)*var_Ybar0+(3/4)*var_Ybar1+2*cov_Ybar0))

. di %8.7f se_ate
0.9766259
. 
. restore 
\end{lstlisting}


Assuming that 4 out of 7 clusters are assigned to treatment, the standard error of the ATE will be 0.98.

\item Why do the two methods of forming clusters lead to different standard errors? What are the implications for the design of cluster randomized experiments?\\
Answer:\\
The first method clusters the most similar villages together, and the second method clusters the most dissimilar villages together. As a result, the variances of the average within-cluster potential outcomes are much larger in the first method and smaller in the second. As a result, the second method produces a much narrower standard error of the ATE estimate. The implication for clustered design is that the more similar the observations with a cluster, the less precise the estimates we can produce. When possible, cluster heterogeneous observations together.
\end{enumerate}

\section*{Question 12}
Below is a schedule of potential outcomes for six classrooms, which are located in three schools.  Using a cluster randomized design, researchers will assign one of the three schools (and all the classrooms it contains) to the treatment group. [5 points]

% Table generated by Excel2LaTeX from sheet 'Sheet1'
\begin{table}[H]
  \centering
  \caption{Question 12 Table}
    \begin{tabular}{cccc}
    \toprule
    School & Classroom & $Y_i(0)$ & $Y_i(1)$ \\
    \midrule
    A     & A-1   & 0     & 0 \\
    B     & B-1   & 0     & 1 \\
    B     & B-2   & 0     & 1 \\
    C     & C-1   & 0     & 2 \\
    C     & C-2   & 0     & 2 \\
    C     & C-3   & 0     & 2 \\
    \bottomrule
    \end{tabular}%
  \label{tab:addlabel}%
\end{table}%

\begin{enumerate}[a)]
\item What is the average treatment effect among the six classrooms?
\begin{align*}
\frac{2+2+2+1+1+0}{6} = 1.333
\end{align*}

\item There are three possible randomizations. Is the difference-in-means estimator unbiased?\\
Answer:\\
The estimated ATE is 0 if school A is assigned to treatment, 1 if school B is assigned to treatment, and 2 if school C is assigned to treatment. So if we take the average of three estimates the ATE is $\frac{0+1+2}{3}=1 \neq 1.33$ and is therefore biased. When potential outcomes are related to cluster size, cluster randomization is prone to bias in small samples, as in this case. This condition holds in this case: the biggest cluster, Cluster C, has larger than average $Y(1)$ values.
\item In general, cluster random assignment generates biased results when (i) clusters vary in size, (ii) potential outcomes vary by cluster, and (iii) the number of clusters is too small to ensure that m of N units are placed into the treatment condition in each randomization.  Show what happens in this example when School A and School B are combined for purposes of random assignment, so that there is a 0.5 probability that either School C is placed in treatment or Schools A and B are placed in treatment. Does this design yield unbiased estimates?  What are the implications of this exercise for the design of cluster randomized experiments? \\
Answer:\\
If A and B are combined and put into treatment, the estimated ATE is 2/3; if C is treated, the estimated ATE is 2. Therefore, the average estimated ATE is $\frac{2/3+2}{2}=1.33$, which is the true ATE. Therefore, combining clusters to make cluster size constant eliminates bias.  The implication is that bias can be avoided by constructing clusters of equal size.
\end{enumerate}

\end{document}
