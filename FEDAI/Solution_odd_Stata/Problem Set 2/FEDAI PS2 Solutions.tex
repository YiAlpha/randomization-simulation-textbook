% %%%%%%%%%%%%%%%%%%%%%%%%%%%%%%%%%%%%%%%%%%%%%%%%%%%%%%%%%%%%%%%%%%%%%%%%%%%%%%%%%%%%%%%%%%%%
% PROBLEM SET LATEX TEMPLATE FILE
% DEFINE DOCUMENT STYLE, LOAD PACKAGES
\documentclass[11pt,notitlepage]{article}\usepackage[]{graphicx}\usepackage[]{color}
%% maxwidth is the original width if it is less than linewidth
%% otherwise use linewidth (to make sure the graphics do not exceed the margin)
\makeatletter
\def\maxwidth{ %
  \ifdim\Gin@nat@width>\linewidth
    \linewidth
  \else
    \Gin@nat@width
  \fi
}
\makeatother

\definecolor{fgcolor}{rgb}{0.345, 0.345, 0.345}
\newcommand{\hlnum}[1]{\textcolor[rgb]{0.686,0.059,0.569}{#1}}%
\newcommand{\hlstr}[1]{\textcolor[rgb]{0.192,0.494,0.8}{#1}}%
\newcommand{\hlcom}[1]{\textcolor[rgb]{0.678,0.584,0.686}{\textit{#1}}}%
\newcommand{\hlopt}[1]{\textcolor[rgb]{0,0,0}{#1}}%
\newcommand{\hlstd}[1]{\textcolor[rgb]{0.345,0.345,0.345}{#1}}%
\newcommand{\hlkwa}[1]{\textcolor[rgb]{0.161,0.373,0.58}{\textbf{#1}}}%
\newcommand{\hlkwb}[1]{\textcolor[rgb]{0.69,0.353,0.396}{#1}}%
\newcommand{\hlkwc}[1]{\textcolor[rgb]{0.333,0.667,0.333}{#1}}%
\newcommand{\hlkwd}[1]{\textcolor[rgb]{0.737,0.353,0.396}{\textbf{#1}}}%

\usepackage{framed}
\makeatletter
\newenvironment{kframe}{%
 \def\at@end@of@kframe{}%
 \ifinner\ifhmode%
  \def\at@end@of@kframe{\end{minipage}}%
  \begin{minipage}{\columnwidth}%
 \fi\fi%
 \def\FrameCommand##1{\hskip\@totalleftmargin \hskip-\fboxsep
 \colorbox{shadecolor}{##1}\hskip-\fboxsep
     % There is no \\@totalrightmargin, so:
     \hskip-\linewidth \hskip-\@totalleftmargin \hskip\columnwidth}%
 \MakeFramed {\advance\hsize-\width
   \@totalleftmargin\z@ \linewidth\hsize
   \@setminipage}}%
 {\par\unskip\endMakeFramed%
 \at@end@of@kframe}
\makeatother

\definecolor{shadecolor}{rgb}{.97, .97, .97}
\definecolor{messagecolor}{rgb}{0, 0, 0}
\definecolor{warningcolor}{rgb}{1, 0, 1}
\definecolor{errorcolor}{rgb}{1, 0, 0}
\newenvironment{knitrout}{}{} % an empty environment to be redefined in TeX

\usepackage{alltt}    % ADD COMMENTS USING A PERCENT SIGN
\usepackage{amsfonts}
\usepackage{amsthm}
\usepackage{amsmath, booktabs}
\usepackage{mathtools}
\usepackage{amssymb}
\usepackage{subfig}
\usepackage{setspace}
\usepackage{fullpage}
\usepackage{verbatim}
\usepackage{graphicx}
\usepackage{tabularx}
\usepackage{longtable}
\usepackage{multicol}
\usepackage{multirow}
\setlength{\parindent}{0in}		% uncomment to remove indent at start of paragraphs
\usepackage{pdflscape}
\usepackage[english]{babel}
\usepackage[pdftex]{hyperref}
\usepackage{natbib}
\usepackage{caption}
\usepackage{amsmath}
\usepackage{amsfonts}
\usepackage{graphics}
\usepackage{multirow}
\usepackage{graphics}
\usepackage{hyperref}
\usepackage{longtable}
\usepackage{latexsym}
\usepackage{rotating}
\usepackage{setspace}
\usepackage{layouts} 
\usepackage[titletoc]{appendix}
\DeclareGraphicsExtensions{.pdf,.jpg,.png}
\usepackage[margin=1in]{geometry}
\usepackage{enumerate}
\usepackage{float}

\usepackage{xcolor}
\usepackage[printwatermark]{xwatermark}



\newcolumntype{L}[1]{>{\raggedright\let\newline\\\arraybackslash\hspace{0pt}}m{#1}}
\newcolumntype{C}[1]{>{\centering\let\newline\\\arraybackslash\hspace{0pt}}m{#1}}
\newcolumntype{R}[1]{>{\raggedleft\let\newline\\\arraybackslash\hspace{0pt}}m{#1}}
% FONTS
\usepackage[T1]{fontenc}					% always use this no matter what

% uncomment any one of these to see what it does to your font!
%\usepackage{pxfonts}
%\usepackage{cmbright}
%\usepackage{txfonts}
%\usepackage[adobe-utopia]{mathdesign}
%\usepackage{kpfonts}
%\usepackage{lmodern}
%\usepackage{newtxtext,newtxmath}



\IfFileExists{upquote.sty}{\usepackage{upquote}}{}
\begin{document}
\title{Field Experiments: Design, Analysis and Interpretation \\
Solutions for Chapter 2 Exercises}
\author{Alan S. Gerber and Donald P. Green\footnote{Solutions prepared by Peter M. Aronow and revised by Alexander Coppock}}
\date{\vspace{-5ex}}
\maketitle


\section*{Question 1}
Potential outcomes notation:[5 points]

\begin{enumerate}[a)]
\item Explain the notation ``$Y_{i}(0)$.''\\
Answer:\\
The potential outcome for subject $i$ if this subject were untreated. Another way to put it: the untreated potential outcome for subject $i$. Note that the argument in parentheses refers to the case in which d (the treatment indicator) equals zero (lack of treatment).

\item Explain the notation ``$Y_{i}(0)|D_i=1$'' and contrast it with the notation ``$Y_{i}(0)|d_i=1$''\\
Answer:\\
$Y_{i}(0)|D_i=1$ The untreated potential outcome for subject $i$ who hypothetically receives the treatment, whereas $Y_{i}(0)|d_i=1$ is the untreated potential outcome for subject $i$ if $i$ is actually treated.

\item Contrast the meaning of ``$Y_{i}(0)$'' with the meaning of ``$Y_{i}(0)|D_{i}=0$.''\\
Answer:\\
The first is the untreated potential outcome for subject i; the second is the untreated potential outcome for a subject who is untreated under some hypothetical assignment.

\item Contrast the meaning of ``$Y_{i}(0)|D_{i}=1$'' with the meaning of ``$Y_{i}(0)|D_{i}=0$.'' \\
Answer:\\
The first is the untreated potential outcome for a subject in the treatment group under a hypothetical treatment allocation; the second is the untreated potential outcome for a subject who is in the control group under a hypothetical allocation.

\item Contrast the meaning of $E[Y_i(0)]$ with the meaning of $E[Y_i(0) | D_{i}=1]$.\\
Answer: The first is the expectation of the untreated potential outcome for the entire subject pool, whereas the second is the expected untreated potential outcome for a randomly selected subject who would receive the treatment in a hypothetical allocation.

\item Explain why the ``selection bias'' term in equation (2.15), $E[Y_{i}(0)|D_{i}=1]-E[Y_{i}(0)|D_{i}=0]$, is zero when $D_{i}$ is randomly assigned. \\
Answer:\\
This equality states that when treatments are allocated randomly, the untreated potential outcome for a subject who actually receives the treatment is, in expectation, the same as the untreated outcome for a subject who goes untreated.  This equality follows from the fact that under random assignment, $E[Y_{i}(0)|D_{i}=1]=E[Y_{i}(0)]$ and $E[Y_{i}(0)|D_{i}=0]=E[Y_{i}(0)]$, since both the treatment and control groups are random samples of the entire set of potential outcomes.

\end{enumerate}

\section*{Question 2}
\begin{knitrout}
\definecolor{shadecolor}{rgb}{0.969, 0.969, 0.969}\color{fgcolor}\begin{kframe}
\begin{verbatim}





\end{verbatim}
\end{kframe}
\end{knitrout}

\section*{Question 3}
Use the values depicted in Table 2.1 to complete the following table.[5 points]
\begin{enumerate}[a)]
\item Fill in the number of observations in each of the nine cells. \\
see below.
\item Indicate the percentage of all subjects that fall into each of the nine cells. (These cells represent what is known as the joint distribution of $Y_{i}(0)$ and $Y_{i}(1)$, or $p(Y_{i}(0), Y_{i}(1))$.\\
see below.
\item At the bottom of the table, indicate the proportion of subjects falling into each category of $Y_{i}(1)$ (These cells represent what is known as the marginal distribution of $Y_{i}(1)$, or $p(Y_{i}(1))$. \\
see below.
\item At the right of the table, indicate the proportion of subjects falling into each category of $Y_{i}(0)$ (i.e., the marginal distribution of $Y_{i}(0)$, or $p(Y_{i}(0))$. \\

\begin{table}[H]
  \centering
  \caption{Table for Question 3}
    \begin{tabular}{r|r|c|c|c|c|}
     \multicolumn{1}{c}{}       &      \multicolumn{1}{c}{}    & \multicolumn{3}{c}{$Y_{i}(1)$}  &    \multicolumn{1}{c}{} \\ \cline{2-6}
     &        &  \multicolumn{1}{c}{15}    &  \multicolumn{1}{c}{20}    &    30    &   \\ \cline{2-6}
   \multirow{3}[0]{*}{$Y_{i}(0)$} & 10    & 1: 1/7   & 1: 1/7   & 0: 0/7     & 2/7 \\\cline{3-6}
     & 15    & 2: 2/7   & 0: 0/7     & 1: 1/7   & 3/7 \\ \cline{3-6}
     & 20    & 1: 1/7   & 0: 0/7     & 1: 1/7   & 2/7 \\ \cline{2-6}
          &       & 4/7   & 1/7   & 2/7   & 1 \\
\cline{2-6}
    \end{tabular}
\end{table}

\item Use the table to calculate the conditional expectation that $E[Y_{i}(0)|Y_{i}(1) > 15]$.  (Hint: this expression refers to the expected value of $Y_i (0)$ given that $Y_i (1)$ is greater than 15.) 

\begin{align*}
E[Y_i (0)| Y_i (1)>15] & =\sum_i Y_i(0) \frac{pr(Y(0)=Y_i (0),Y_i (1)>15)}{pr(Y_i (1)>15)} \\
&=10*\frac{(1/7)}{(3/7)} +15* \frac{(1/7)}{(3/7)}+20*\frac{(1/7)}{(3/7)}\\
&=15
\end{align*}

\item Use the table to calculate the conditional expectation that $E[Y_{i}(1)|Y_{i}(0) > 15]$.  

\begin{align*}
E[Y_i (1)| Y_i (0)>15] & =\sum_i Y_i(1) \frac{pr(Y(1)=Y_i (1),Y_i (0)>15)}{pr(Y_i (0)>15)} \\
&=15*\frac{(1/7)}{(2/7)} +20* \frac{0}{(2/7)}+30*\frac{(1/7)}{(2/7)}\\
&=22.5
\end{align*}


\end{enumerate}  


\section*{Question 4}
\begin{knitrout}
\definecolor{shadecolor}{rgb}{0.969, 0.969, 0.969}\color{fgcolor}\begin{kframe}
\begin{verbatim}





\end{verbatim}
\end{kframe}
\end{knitrout}

\section*{Question 5}
A researcher plans to ask six subjects to donate time to an adult literacy program. Each subject will be asked to donate either 30 or 60 minutes.  The researcher is considering three methods for randomizing the treatment.  One method is to flip a coin before talking to each person and to ask for a 30-minute donation if the coin comes up heads or a 60-minute donation if it comes up tails. The second method is to write ``30'' and ``60'' on three playing cards each, and then shuffle the six cards. The first subject would be assigned the number on the first card, the second subject would be assigned the number on the second card, and so on.  A third method is to write each number on three different slips of paper, seal the six slips into envelopes, and shuffle the six envelopes before talking to the first subject.  The first subject would be assigned the first envelope, the second subject would be assigned the second envelope, and so on. [10 points]

\begin{enumerate}[a)]
\item Discuss the strengths and weaknesses of each approach.\\
Answer:\\
All three physical methods of random assignment require that the person or persons in charge of implementing the randomization follow the intended protocol: dice must be rolled once per subject, and cards or envelopes must be shuffled thoroughly. Assuming that the mechanics of each physical method of randomization are carried out, the limitation of the dice method is that possibility that the allocation of treatments could wind up being imbalanced; in principle, one could flip a coin 6 times and come up with 6 heads, in which case the treatments would not vary. The card method overcomes this problem and ensures that exactly half of the subjects will receive each treatment. The advantage of the sealed envelope method over the card method is the fact that envelopes help prevent the person who is allocating subjects from deliberately or unconsciously exercising discretion over who receives which treatment, thereby subverting the randomization.  It also prevents the implementer from anticipating the next treatment assignment (until the last few envelopes).
\item In what ways would your answer to (a) change if the number of subjects were 600 instead of 6?  \\
Answer:\\
As the N increases, the dice method becomes more likely to produce a 50-50 division in treatments. For example, with 600 subjects, the probability of obtaining an assignment as imbalanced as 250-350 is less than 1-in-10,000. 
\item What is the expected value of D if the coin toss method is used?  What is the expected value of D if the sealed envelope method is used? \\
Answer:\\
The methods produce identical results, in expectation. \\
The expected value of X if the dice is used: $E[x_{dice}]=\frac{1}{2} 30+ \frac{1}{2} 60=45$.  \\
The expected value of X if the envelope method is used: $E[x_{envelope}]=\frac{30+30+30+60+60+60}{6}=45$\\
\end{enumerate}

\section*{Question 6}
\begin{knitrout}
\definecolor{shadecolor}{rgb}{0.969, 0.969, 0.969}\color{fgcolor}\begin{kframe}
\begin{verbatim}





\end{verbatim}
\end{kframe}
\end{knitrout}

\section*{Question 7}
Suppose that an experiment were performed on the villages in Table 2.1, such that two villages are allocated to the treatment group and the other five villages to the control group. Suppose that an experimenter randomly selects villages 3 and 7 from the set of seven villages and places them into the treatment group.  Table 2.1 shows that these villages have unusually high potential outcomes. [10 points]
\begin{enumerate}[a)]
\item Define the term \textit{unbiased estimator}.\\
Answer:\\
An unbiased estimator is a formula that, on average over hypothetical replications of the study, generates estimates that equal the true parameter. Any given estimate may be too high or too low, but on average over hypothetical replications of the study, an unbiased estimator recovers the estimand.
\item Does this allocation procedure produce upwardly biased estimates?  Why or why not? \\
Answer:\\
No.  The procedure is unbiased because the two villages selected for treatment as drawn randomly from the list of villages; therefore their potential outcomes are, in expectation, identical to the average potential outcomes for the entire set of villages.  Although in this instance the random allocation procedure produced an estimate that was not equal to the true ATE, the procedure remains unbiased because across all possible random allocations, the average estimate equals the true ATE.
\item Suppose that instead of using random assignment, the researcher placed Villages 3 and 7 into the treatment group because the treatment could be administered inexpensively in those villages.  Explain why this procedure is prone to bias. \\
Answer:\\
Unlike random assignment, inexpensiveness is not a criterion the ensures that the treatment group and control group have potential outcomes that are identical in expectation. For example, it may be that villages are inexpensive to treat because they are near transportation networks, which may in turn mean that their potential outcomes are unusual due to increased access to or demand for water sanitation.
\end{enumerate}

\section*{Question 8}
\begin{knitrout}
\definecolor{shadecolor}{rgb}{0.969, 0.969, 0.969}\color{fgcolor}\begin{kframe}
\begin{verbatim}





\end{verbatim}
\end{kframe}
\end{knitrout}

\section*{Question 9}
A researcher wants to know how winning large sums of money in a national lottery affects people's views about the estate tax.  The researcher interviews a random sample of adults and compares the attitudes of those who report winning more than \$10,000 in the lottery to those who claim to have won little or nothing. The researcher reasons that the lottery chooses winners at random, and therefore the amount that people report having won is random. [10 points]

\begin{enumerate}[a)]
\item Critically evaluate this assumption. (Hint: are the potential outcomes of those who report winning more than \$10,000 identical, in expectation, to those who report winning little or nothing?)  \\
Answer:\\
This assumption may not be plausible in this application.  Although lottery winners are chosen at random from the pool of players in a given lottery, this study does not compare (randomly assigned) winners and losers from a pool of lottery players.  Instead, winners are compared to non-winners, where the latter group may include non-players.  Winning is therefore not randomly assigned.  If frequent players are more likely to win than non-players and the two groups have different potential outcomes, the comparison of the two groups may be prone to bias.
\item Suppose the researcher were to restrict the sample to people who had played the lottery at least once during the past year.  Is it now safe to assume that the potential outcomes of those who report winning more than \$10,000 are identical, in expectation, to those who report winning little or nothing?  \\
Answer:\\
The assumption is not rooted in a randomization procedure because frequent players are still more likely to be winners than infrequent players.  Unfortunately, without detailed information about how many tickets were purchased for each lottery, we don't know the exact probability that each subject would win.  If frequent and infrequent players have different potential outcomes, the comparison is prone to bias (although, arguably, less bias than a comparison of winners to non-players).

\end{enumerate}



\section*{Question 10}
\begin{knitrout}
\definecolor{shadecolor}{rgb}{0.969, 0.969, 0.969}\color{fgcolor}\begin{kframe}
\begin{verbatim}





\end{verbatim}
\end{kframe}
\end{knitrout}


\section*{Question 11}
Several randomized experiments have assessed the effects of drivers' training classes on the likelihood that a student will be involved in a traffic accident or receive a ticket for a moving violation.   A complication arises because students who take drivers' training courses typically obtain their licenses faster than students who do not take a course. (The reason is unknown but may reflect the fact that those who take the training are better prepared for the licensing examination.)  If students in the control group on average start driving much later, the proportion of students who have an accident or receive a ticket could well turn out to be higher in the treatment group.  Suppose a researcher were to compare the treatment and control group in terms of the number of accidents that occur within 3 years of obtaining a license.[10 points]

\begin{enumerate}[a)]
\item Does this measurement approach maintain symmetry between treatment and control groups?  \\
Answer:\\
No, because the measurement procedure differs for treatment and control groups.  If control subjects tend to receive their licenses later, the apparent treatment effect may be biased by the fact that the control group is on average older than the treatment group during the period of study.  If the groups have different ages, their potential outcomes may differ as well.\\
\item Would symmetry be maintained if the outcome measure were the number of accidents per mile of driving?  \\
Answer:\\
No, the problem of asymmetry remains.  The control group tends to be older, so their driving patterns may differ, which in turn implies different potential outcomes.
\item Suppose researchers were to measure outcomes over a period of three years starting the moment at which students were randomly assigned to be trained or not.  Would this measurement strategy maintain symmetry?  Are there drawbacks to this approach? \\
Answer:\\
Yes, this approach maintains symmetry, since the clock starts at the same moment for both treatment and control.  However, the estimand is now the combined effect of the program on the amount of driving and the quality of the drivers. The program might improve driver quality yet produce more accidents due to increased driving.  Some of the uncertainty of interpretation would be eliminated if the driving program were to focus solely on those who already have their licenses, so that eligibility to drive were held constant.
\end{enumerate}


\section*{Question 12}
\begin{knitrout}
\definecolor{shadecolor}{rgb}{0.969, 0.969, 0.969}\color{fgcolor}\begin{kframe}
\begin{verbatim}





\end{verbatim}
\end{kframe}
\end{knitrout}







\end{document}
